%**************************************************************%
%                     File Name: main.tex                      %
%                                                              %
%   International Journal of Offshore and Polar Engineering    %
%               Paper Number is : ISOPE-HE-2011                %
%                                                              %
%              Using Template Prepared by Masashi KASHIWAGI    %
%**************************************************************%
%\documentclass[11pt,fleqn,letterpaper]{article}
 \documentclass[11pt,fleqn,a3]{article}
\usepackage{graphicx} %%%%%%%%%%%%%%% for inserting EPS pictures
\usepackage{amsmath}% AMS math equation
\usepackage{txfonts}
\usepackage{bm}
%
%**************************************************************%
%                   File Name: deffile.tex                     %
%                                                              %
%  This LaTeX file is prepared for writing the manuscript for  %
%    the 12th International Offshore and Polar Engineering     %
%   Conference to be held from May 26-31, 2002 in Kitakyushu   %
%      Organized by ISOPE and hosted by Kyushu University      %
%                                                              %
%           Prepared by Masashi KASHIWAGI on December 13, 2001 %
%**************************************************************%
%
\makeatletter
%%%%%%%%%%%%%%%%%%%%%%% Style of Footer %%%%%%%%%%%%%%%%%%%%%%%%%%%%%%
\def\ps@isopefoot{\let\@mkboth=\@gobbletwo
 \def\@oddhead{}
 \def\@evenhead{}
%\def\@oddfoot{}
%\def\@evenfoot{}
 \def\@oddfoot{\hbox to \textwidth
   {\footnotesize {Paper No.\,2011-\papernumber} \hfil 
   {\firstauthorname} \hfil {Page number:\ \thepage\ of\ \totalpage}}}
 \def\@evenfoot{\hbox to \textwidth
   {\footnotesize {Paper No.\,2011-\papernumber} \hfil 
   {\firstauthorname} \hfil {Page number:\ \thepage\ of\ \totalpage}}}
}%
%---------------------------------------------------------------
\setcounter{topnumber}{3}     %
\def\topfraction{1.0}         %
\setcounter{bottomnumber}{2}  %
\def\bottomfraction{1.0}      %
\setcounter{totalnumber}{4}   %
\def\textfraction{0.0}        %
\def\floatpagefraction{1.0}   %
%
\setcounter{dbltopnumber}{4}
\def\dbltopfraction{1.0}
\def\dblfloatpagefraction{1.0}
%---------------------------------------------------------------
%
\def\leftcapmargin{35pt}
\def\rightcapmargin{15pt}
%
\long\def\@makecaption#1#2{%
 \vskip 10\p@ \footnotesize\baselineskip=4.0mm
 \@tempdimb=\leftcapmargin
 \@tempdima\hsize\advance\@tempdima-\@tempdimb
 \@tempdimb=\rightcapmargin
 \advance\@tempdima-\@tempdimb
 \newbox\tempbox
 \setbox\tempbox=\hbox{#1}
 \setbox\@tempboxa=\hbox{\hskip\leftcapmargin #2\hskip\rightcapmargin}
 \ifdim \wd\@tempboxa >\hsize
 \hbox to\hsize{\hfil \hskip-\the\wd\tempbox \hskip\leftcapmargin #1 
 \parbox[t]\@tempdima{#2}\hskip\rightcapmargin}%
 \else \hbox to\hsize{\hfil #1 #2\hfil}\fi
%\setcounter{subfigure}{0}
%\setcounter{subtable}{0}
}
%--------------------------------------------------------
\def\fnum@figure{{\footnotesize Fig.\,\thefigure}\ }
\def\fnum@table{{\footnotesize Table\,\thetable}\,}
%
%%%%%%%%%%%%%%%%%%%%%%%%%%%%%%%%%%%%%%%%%%%%%%%%%%%%%%%%%%%%%%%%%%%%%%
\def\thesection{}
\def\thesubsection{}
\def\thesubsubsection{}
%
\def\section{\@startsection{section}{1}{\z@}
{2ex plus .3ex minus .2ex}{1.5ex plus .1ex}{\small}}
\def\subsection{\@startsection{subsection}{2}{\z@}
{1.5ex plus .2ex minus .2ex}{1ex plus .1ex}{\small\bf}}
\def\subsubsection{\@startsection{subsubsection}{3}{\z@}
{1.5ex plus .2ex minus .2ex}{1ex plus .1ex}{\footnotesize\it}}
%
\def\paragraph{\@startsection{paragraph}{4}{\z@}
{1ex plus .3ex minus .2ex}{-1em}{\large\bf}}
\def\subparagraph{\@startsection{subparagraph}{4}
{\parindent}{1ex plus .3ex minus .2ex}{-1em}{\small\bf}}
%---------------------------------------------------------------------
\def\eqnarray{%
 \stepcounter{equation}%
 \let\@currentlabel=\theequation
 \global\@eqnswtrue
 \global\@eqcnt\z@
 \tabskip\mathindent
 \let\\=\@eqncr
 $$\halign to \displaywidth\bgroup\@eqnsel\hskip\@centering
 $\displaystyle\tabskip\z@{##}$&\global\@eqcnt\@ne
 \hfil$\displaystyle{{}##{}}$\hfil
 &\global\@eqcnt\tw@$\displaystyle\tabskip\z@{##}$\hfil
 \tabskip\@centering&\llap{##}\tabskip\z@\cr}
%
%%%%%%%%%%%%%%%%%%%%%%%%%%%%%%%%%%%%%%%%%%%%%%%%%%%%%%%%%%%%%%%%%%%%%%
%
\def\@cite#1#2{\,[{\hbox{#1\if@tempswa , #2\fi}}]}
%
\def\thebibliography#1{%
\begin{flushleft}\section*{REFERENCES}\end{flushleft}
\list
{[\arabic{enumi}]}{\settowidth\labelwidth{#1)}\leftmargin\labelwidth
 \advance\leftmargin\labelsep
 \usecounter{enumi}}
 \def\newblock{\hskip .11em plus .33em minus .07em}
 \sloppy
 \sfcode`\.=1000\relax}
\let\endthebibliography=\endlist
%=====================================================================
%
\newif\if@lastpagecolumnalign \@lastpagecolumnaligntrue
% \@lastpagecolumnalignfalse  % if you do NOT need the function No.3
%
\if@lastpagecolumnalign %============ FUNCTION No.3 ============
%
\newdimen\lastp@geheight \lastp@geheight=10mm
\newsavebox{\lastp@gebox}
%
% define \endabstract
%
\long\def\endabstract#1{\gdef\@endabstract{#1}} % permit paragraphs
%
% to create enough space of footnote to align columns of the last page
%
\def\lastpagecontrol{\@ifnextchar [{\l@stpagecontrol}%
 {\l@stpagecontrol[\z@]}}
%
\def\l@stpagecontrol[#1]#2{\global\lastp@geheight=#2%
%
 \@ifundefined{maxsize}{}{\global\advance\maxsize-#2}% for supertab.sty
%
 \@ifundefined{@endabstract}{}{% save \endabstract in a box
 \global\sbox{\lastp@gebox}{%
  \begin{minipage}{\textwidth}\vspace*{#1}%
  \hrule width \textwidth \vspace{1ex}%
  \begin{list}{}{\setlength{\leftmargin}{0.15\textwidth}%
  \listparindent=10pt \parsep=0pt
  \topsep=0pt \partopsep=0pt
  \setlength{\rightmargin}{\leftmargin}\small}%
  \item \ignorespaces\hspace*{\listparindent}\ignorespaces
                 \@endabstract
  \end{list}%
  \vspace{1ex} \hrule width \textwidth
  \end{minipage}}}%
%
% insert footnote with abstract if enough room left
%                 without abstract otherwise
%
  \@tempdima\ht\lastp@gebox \advance\@tempdima\dp\lastp@gebox
 \ifdim\@tempdima>\lastp@geheight
  \@tempdima\lastp@geheight \global\lastp@geheight=0pt
 \else
  \global\advance\lastp@geheight -\@tempdima
  \@tempdima\lastp@geheight \global\lastp@geheight\textheight
 \fi
  \def\footnoterule{\null}% force it to \null at the last page
  \insert\footins{\footnotesize
  \interlinepenalty\interfootnotelinepenalty
  \splittopskip\footnotesep
  \splitmaxdepth \dp\strutbox \floatingpenalty \@MM
  \hsize\textwidth \@parboxrestore
  \ifdim\lastp@geheight=\z@\else\usebox{\lastp@gebox}\fi%
  \vspace*{\@tempdima}}}
%
% set the layout of the last line of the document
%
\def\lastpagesettings{\@ifnextchar [{\l@stpagesettings}%
 {\l@stpagesettings[\z@]}}
%
\def\l@stpagesettings[#1]{%
%\begin{flushright}%
% {\footnotesize(`{\tt\jobname.tex}'~~@~\today)}%
%\end{flushright}%
%
% put \endabstract in the next page when no room left for it
%
 \ifdim\lastp@geheight=\z@
 \onecolumn\null\vspace*{#1}\noindent\usebox{\lastp@gebox}%
 \fi}
%
\fi %===========================================================
%
\makeatother
%---------------------------------------------------------------
\def\jsection#1{\section{\hspace*{-4.0mm}\uppercase{#1}}}
\def\jsubsection#1{\subsection{\hspace*{-4.0mm}#1}}
\def\jsubsubsection#1{\subsubsection{\hspace*{-4.0mm}#1}}
%---------------------------------------------------------------
%%%%%%%%%%%%%%%%%%%%%%%%%%%%%%%%%%%%%%%%%%%%%%%%%%%%%%%%%%%%%%%%
\autospacing
\newcommand{\bdm}{\begin{displaymath}}
\newcommand{\edm}{\end{displaymath}}
\newcommand{\be}{\begin{equation}}
\newcommand{\ee}{\end{equation}}
\newcommand{\bea}{\begin{eqnarray}}
\newcommand{\eea}{\end{eqnarray}}
% 
\newcommand{\bc}{\begin{center}}
\newcommand{\ec}{\end{center}}
\newcommand{\bfl}{\begin{flushleft}}
\newcommand{\efl}{\end{flushleft}}
\newcommand{\bfr}{\begin{flushright}}
\newcommand{\efr}{\end{flushright}}
\newcommand{\ds}{\displaystyle}
%---------------------------------------------------------------
\newcommand{\mbf}[1]{\mbox{\boldmath$#1$}}
\newcommand{\ubar}[1]{\overline{#1}}
\newcommand{\dint}{\int\!\!\!\int}
\newcommand{\tint}{\int\!\!\!\int\!\!\!\int}
\newcommand{\heikou}{\,/\!/\,}
\newcommand{\hs}{\hspace*{0.2mm}}
\newcommand{\iomega}{i\hspace*{0.25mm}\omega\hspace*{0.2mm}}
\newcommand{\jneqi}{j = \hspace*{-1.6mm}/\hspace*{0.5mm} i}
\newcommand{\rmP}{{\rm P}}
\newcommand{\rmQ}{{\rm Q}}
\newcommand{\cint}{\int_0^\infty \hspace*{-6.2mm}C\hspace*{3.8mm}}
\newcommand{\e}{\,e}
\newcommand{\noteq}{=\hspace*{-3.0mm}/\hspace*{1.25mm}}
\newcommand{\ch}{\hspace*{0.5mm}{\rm ch}\hspace*{0.3mm}}
\newcommand{\sh}{\hspace*{0.5mm}{\rm sh}\hspace*{0.3mm}}
\newcommand{\dash}{{\hs\prime}}
\newcommand{\npar}{\par\vspace*{10pt}}
%
%%%%%%%%%%%%%%%% Upper case of Greek Letters %%%%%%%%%%%%%%%%%%
\newcommand{\gPi}{\mathnormal{\Pi}}
\newcommand{\gGamma}{\mathnormal{\Gamma}}
\newcommand{\gPhi}{\mathnormal{\Phi}}
\newcommand{\gPsi}{\mathnormal{\Psi}}
\newcommand{\gTheta}{\mathnormal{\Theta}}
\newcommand{\gDelta}{\mathnormal{\Delta}}
\newcommand{\gXi}{\mathnormal{\Xi}}
\newcommand{\gLambda}{\mathnormal{\Lambda}}
\newcommand{\gOmega}{\mathnormal{\Omega}}
%
%%%%%%%%%%%%%%%%%%%%%%%%%%%%%%%%%%%%%%%%%%%%%%%%%%%%%%%%%%%%%%%
\setlength{\textheight}  {240.0mm}
\setlength{\textwidth}   {191.9mm}
\setlength{\columnsep}   {8.9mm}
\setlength{\parindent}   {0.0mm}
\setlength{\mathindent}  {0.0mm}
%
\newlength{\mybaselineskip}
\setlength{\mybaselineskip}{11pt}
\newlength{\myleftmargin}
\newlength{\mytopmargin} 
\setlength{\myleftmargin}  {11.0mm}
\setlength{\mytopmargin}   {13.0mm}
%%%%%% For the letter size (ISOPE Format) %%%%%%%%%%%%%
\setlength{\myleftmargin}  {12.0mm}
\setlength{\mytopmargin}   { 3.0mm}
%%%%%%%%%%%%%%%%%%%%%%%%%%%%%%%%%%%%%%%%%%%%%%%%%%%%%%%
\setlength{\oddsidemargin} {-1in}
\setlength{\topmargin}     {-1in}
\addtolength{\oddsidemargin}{\myleftmargin}
\addtolength{\topmargin}    {\mytopmargin}
\setlength{\headheight}  { 0.0mm}
\setlength{\footskip}    {20.0mm}
%
\sloppy


\pagestyle{empty}
%\pagestyle{isopefoot} 
%
%%%%%%%%%%%%%%%%%%%%%%%%%%%%%%%%%%%%%%%%%%%%%%%%%%%%%%%%%%%%%%%%%%
%%%%%%%%%%%%%%%%%%% Write below for the footer %%%%%%%%%%%%%%%%%%%
%%%%%%%%%%%%%%%%%%%%%%%%%%%%%%%%%%%%%%%%%%%%%%%%%%%%%%%%%%%%%%%%%%
\newcommand{\papernumber} {TPC-970}
\newcommand{\firstauthorname} {Masashi Kashiwagi}
\newcommand{\totalpage} {7}
%
%+-+-+-+-+-+-+-+-+-+-+-+-+-+-+-+-+-+-+-+-+-+-+-+-+-+-+-+-+-+-+-+-+
\begin{document}
\baselineskip \mybaselineskip
\twocolumn[ \par\vspace*{40.0mm}\large\bc{\bf
%
%%%%%%%%%%%%%%%%%%%%%%%%%%%%%%%%%%%%%%%%%%%%%%%%%%%%%%%%%%%%%%%%%%
%%%%%%%%%%%%%%%%%%%%%%%%% Title of Paper %%%%%%%%%%%%%%%%%%%%%%%%%
%%%%%%%%%%%%%%%%%%%%%%%%%%%%%%%%%%%%%%%%%%%%%%%%%%%%%%%%%%%%%%%%%%
Hydrodynamic Study on Added Resistance by Means of\\%\\ %%% 1st Line
Unsteady Wave Analysis Method %%% 2nd Line if any
}\ec %
%%%%%%%%%%%%%%%%%%%%%%%%%%%%%%%%%%%%%%%%%%%%%%%%%%%%%%%%%%%%%%%%%%
%%%%%%%%%%%%%%%%% Name of Author(s) and Affiliation %%%%%%%%%%%%%%
%%%%%%%%%%%%%%%%%%%%%%%%%%%%%%%%%%%%%%%%%%%%%%%%%%%%%%%%%%%%%%%%%%
\par\vspace{1mm}\footnotesize\bc{\small\it
%
Masashi Kashiwagi, Takuma Sasakawa, Tomoki Wakabayashi
}\\ %%%%+-+-+-+-+-+-+ Write the Affiliation below +-+-+-+-+-+-+
Department of Naval Architecture and Ocean Engineering, Osaka 
University, Osaka, Japan
%
%%%%%%%%%%%%%%%%%%%%%%%%%%%%%%%%%%%%%%%%%%%%%%%%%%%%%%%%%%%%%%%%%%
\ec\par\vspace*{15mm}
%%%%%%%%%%%%%%%%%%%%%%%%%%%%%%%%%%%%%%%%%%%%%%%%%%%%%%%%%%%%%%%%%%
]
\footnotesize\par\noindent{\small ABSTRACT}\par\vspace*{2.0mm}
\baselineskip \mybaselineskip
%%%%%%%%%%%%%%%%%%%%%%%%%%%%%%%%%%%%%%%%%%%%%%%%%%%%%%%%%%%%%%%%%%
%%%%%%%%%%%%%%%%%%%%%%% Write Abstract Here %%%%%%%%%%%%%%%%%%%%%%
%%%%%%%%%%%%%%%%%%%%%%%%%%%%%%%%%%%%%%%%%%%%%%%%%%%%%%%%%%%%%%%%%%
%
It is known that the added resistance in waves can be computed from 
ship-generated unsteady waves through the unsteady wave analysis method.
In order to investigate which component or which part of ship-generated 
unsteady waves is dominant in the added resistance, measurements of 
unsteady waves are carried out 
for three canonical problems of wave diffraction, forced oscillations 
in heave and pitch, and free response of ship motions in head waves.
Then the unsteady wave analysis using the 
Fourier transform of measured waves along a line parallel to 
the ship's movement is performed to compute the added resistance.
With these analyzed results and numerical computations by a 
unified slender-ship theory, 
validity of the linear superposition of 
component waves is studied.
It is found that the added resistance due to local nonlinear 
wave generation with energy dissipation is rather large when 
the ship-motion amplitude becomes large and this nonlinearity is
prominent near the fore-front part of the wave.
%-----------------------------------------------------------------
\footnotesize
\par\bigskip\noindent{{\small KEY\,\,WORDS}}:
%%%%%%%%%%%%%%%%%%%%%%%%%%%%%%%%%%%%%%%%%%%%%%%%%%%%%%%%%%%%%%%%%%
%%%%%%%%%%%%%%%%%%%%%%% Write Key Words %%%%%%%%%%%%%%%%%%%%%%%%%%
%%%%%%%%%%%%%%%%%%%%%%%%%%%%%%%%%%%%%%%%%%%%%%%%%%%%%%%%%%%%%%%%%%
Added resistance; unsteady wave analysis; Fourier transform; 
linear superposition; nonlinear effects.
%
\par\vspace*{2mm}%
%%%%%%%%%%%%%%%%%%%%%%%%%%%%%%%%%%%%%%%%%%%%%%%%%%%%%%%%%%%%%%%%%%
%%%%%%%%%%%%%%%%%%%%%%% Start Introduction %%%%%%%%%%%%%%%%%%%%%%%
%%%%%% Please use \jsection{ .... } or \section*{ ..... } %%%%%%%%
%%%%%%%%%%%%%%%%%%%%%%%%%%%%%%%%%%%%%%%%%%%%%%%%%%%%%%%%%%%%%%%%%%
\jsection{Introduction}
%-----------------------------------------------------------------
When a ship advances in waves, the resistance on the ship generally increases 
as compared to that in calm sea. This increase of resistance is called 
the added resistance which is crucial in predicting the speed loss of 
a ship in actual seas.
Thus a large amount of work on the added resistance has been made so far. 
Since a pioneering work of Maruo (1960), it has been well recognized that 
the dominant component in the added resistance is the one due to 
unsteady disturbance waves by a ship and their interaction with incident 
waves. Nevertheless, details of the hydrodynamic relation 
between the added resistance and ship-generated unsteady waves 
seem to be still unclear, because most comparisons have been made between 
the total increase in the ship resistance measured by a dynamometer 
in waves and the calculated value with a simplified potential-flow 
theory (e.\hs g. Kashiwagi et al., 2010).
\\

Ohkusu (1980) proposed a method for measuring ship-generated unsteady waves 
and then evaluating the wave amplitude function and the added resistance.
This analysis method enables a comparison at the level of wave 
profile and thus may provide us with deeper understanding on hydrodynamic relations.
However, accurate measurement of unsteady waves is not so easy and 
subsequent analyses for the Fourier transform of wave elevation and for 
the added resistance have not been made in a convincing manner.
\\

In the present study, in order to evaluate accurately the magnitude of 
unsteady wave-making resistance and to understand hydrodynamic relations 
with ship disturbance waves (for instance, which component or which part 
of unsteady waves is dominant in the added resistance), we conducted the 
experiment for the unsteady wave analysis.
In fact, we did the experiment twice, using a modified Wigley model.
In the first experiment, in addition to direct measurement of the 
added resistance by a dynamometer, ship-generated unsteady waves were measured 
only for the motion-free case (where surge, heave, and pitch are free) 
in head waves.
%
Corresponding numerical computations were also implemented with 
Enhanced Unified Theory (EUT) developed by Kashiwagi (1995).
Then not only measured waves but also computed ones were used to validate 
the unsteady wave analysis method for predicting the added resistance 
and to study the effects of local wave and lateral distance for the 
wave measurement (Kashiwagi, 2010).
\\

It was revealed from a comparison in this first experiment 
that a large discrepancy exists between 
the results of added resistance by the direct measurement and by the 
unsteady wave analysis, particularly near the peak of the added resistance where 
wave-induced ship motions also become large (Wakabayashi et al., 2010).
%
Another prominent discrepancy observed in a comparison of the wave profile 
was that the short-wavelength components in measured waves were very small 
in amplitude as opposed to numerical results by EUT.
\\

%---------------------------------------------------------------------
\begin{figure*}[bth]% 0 0 798 187
\footnotesize
\bc
\includegraphics[width=0.8\linewidth]{./figure/fig001.eps}
\ec
%
\par\vspace*{-6mm}
\caption{Coordinate system and schematic illustration 
of wave components} \label{fig001} 
\par\vspace*{-1mm}
\end{figure*}
%---------------------------------------------------------------------


In order to study possible reasons of these discrepancies, the second experiment 
using the same modified Wigley model was conducted as the additional one, 
measuring the unsteady waves for three canonical cases of the diffraction 
problem (where all modes of motion are fixed except for the steady 
translation) in regular head waves, the forced oscillation problem 
in heave and pitch (where surge is fixed) in otherwise calm water, 
and the motion-free problem (where surge, heave, and pitch are free) 
in regular head waves, in terms of larger number of wave probes.
%
Then with these measured results together with analytical and numerical studies, 
discussions are made on the validity of linear superposition of the 
scattering and radiation waves, the amount of nonlinearity in the wave 
generation due to ship motions with larger amplitude, 
and which part of unsteady waves is dominant 
in predicting the added resistance from the wave analysis method.

%
%
\medskip%%%%%%%%%%%%%%%%%%%%%%%%%%%%%%%%%%%%%%%%%%%%%%%%%%%%%%%%%%%%%%
%%%%%%%%%%%%%%%%%%%%%%%%%%% Next Section %%%%%%%%%%%%%%%%%%%%%%%%%%%%%
%%%%%%%% Please use \jsection{ .... } or \section*{ ..... } %%%%%%%%%%
%%%%%%%%%%%%%%%%%%%%%%%%%%%%%%%%%%%%%%%%%%%%%%%%%%%%%%%%%%%%%%%%%%%%%%
\jsection{Added Resistance and Unsteady Waves}
%---------------------------------------------------------------------
We consider a ship advancing at constant forward speed $U$ into 
a regular incident wave of amplitude $A$, circular frequency $\omega_0$. 
The depth of water is assumed infinite; thus the wavenumber of incident wave 
is given by $k_0 =\omega_0^2 /g$, with $g$ the acceleration due to gravity.
Corresponding to the experiment, only the head wave is considered, 
and the analysis is made with a right-hand Cartesian coordinate 
system $O$-$x\hs y\hs z$, with the origin placed at the center of a ship 
and on the undisturbed free surface, which translates with the same constant 
speed as that of a ship along the positive $x$-axis. 
The $z$-axis is positive downward. The unsteady ship responses and ambient 
unsteady flow of fluid are assumed to be linear and periodic with circular 
frequency of encounter $\omega =\omega_0 + k_0\hs U$.
\\

Assuming the flow inviscid with irrotational motion, the velocity potential 
is introduced and written in the form
\be
\gPhi (\mbf{x}, t) = -U\hs x 
+{\rm Re}\,\Big[\,\big\{\hs \phi_0 (\mbf{x}) + 
\phi(\mbf{x})\hs\big\}\e^{\hs\iomega t}\hs \Big]\,,    \label{eq001}
\ee
where $\mbf{x} =(x, y, z)$ and $\phi_0$ denotes the incident-wave potential 
and $\phi$ the disturbance potential.
%
By linear assumption, the disturbance potential $\phi$ is decomposed 
as follows:
\be
\phi (\mbf{x}) = \ds\frac{g A}{\iomega_0} \phi_7 (\mbf{x})
+ \ds\sum_{j=1,3,5} \iomega X_j\hs\ell_j\hs \phi_j (\mbf{x})\,. \label{eq002}
\ee
%
Here $\phi_7$ denotes the scattering potential and $\phi_j$ the 
radiation potential due to the $j$-th mode of motion ($j=1$, 3, 5 for 
surge, heave, and pitch, respectively) with $X_j$ its complex amplitude.
Symbol $\ell_j$ is adopted to express the length dimension for pitch; 
that is, $\ell_5 =L/2$ and $\ell_j=1$ for surge and heave.
\\

The elevation of ship-generated unsteady wave can be computed in the 
linear theory from 
\be
\zeta (x, y) = \ds\frac{1}{g} \bigg(\,\iomega 
-U\frac{\partial}{\partial\hs x}\hs\bigg)\,\phi (x, y, 0)  \label{eq003}
\ee
and each component in the disturbance potential is expressed 
by the far-field representation in EUT as follows:
\be
\phi_j (x, y, 0)= \ds\int_L Q_j (\xi)\,G(x-\xi, y, 0)\, d\xi\,, \label{eq004}
\ee
where $Q_j (x)$ denotes the source strength along the $x$-axis 
and $G (x, y, z)$ the Green function, equivalent to the velocity 
potential due to an oscillating and translating source with 
unit strength.
%
By substituting Eq.\,(\ref{eq002}) and Eq.\,(\ref{eq004}) into Eq.\,(\ref{eq003}) 
and neglecting the local wave term in the Green function, 
the elevation of progressive wave can be computed from 
\bea
\zeta (x, y) = \ds\frac{i}{2\pi} &&
\bigg[ -\!\int_{-\infty}^{\, k_1} \! +  \!\int_{k_2}^{\, k_3} \! 
+\!\int_{k_4}^{\,\infty} \bigg]\,
C(k)\ds\frac{(\omega + kU)}{\omega_0}  \nonumber \\
&& \times \ds\frac{\kappa}{\sqrt{\kappa^2 -k^2}}
\e^{-ikx -i\epsilon_k|\hs y\hs|
\sqrt{\kappa^2 -k^2}}\, dk\,,                    \label{eq005}
\eea
where
\be 
\left. \!\!\! \begin{array}{l}
\kappa =\ds\frac{1}{g} \big(\hs\omega + kU\big)^2
=K +2k\tau +\ds\frac{k^2}{K_0}  \\[2mm]
K=\ds\frac{\omega^2}{g}, \ \tau=\frac{U\omega}{g},\ 
K_0=\frac{g}{U^2},\ \epsilon_k = {\rm sgn} (\omega + kU)
\end{array} \right\}                           \label{eq006}
\ee
\par\vspace*{-3mm}
\bea
&& \left. \!\!\! \begin{array}{l}
k_1 \!\! \\ k_2 \!\! \end{array}\right\} 
=-\ds\frac{K_0}{2}\big(\hs 1+ 2\tau \pm
\sqrt{1+4\tau}\hs\big)\,,                       \label{eq007} \\
&& \left. \!\!\! \begin{array}{l}
k_3 \!\! \\ k_4 \!\! \end{array}\right\} 
=\hspace*{2.0mm} \ds\frac{K_0}{2}\big(\hs 1- 2\tau \mp
\sqrt{1-4\tau}\hs\big)\,,                       \label{eq008}
\eea
\par\vspace*{-3mm}
\bea
&& C(k) =A\hs C_7 (k) -\ds\frac{\omega\hs \omega_0}{g}
\ds\sum_{j=1,3,5} X_j \hs\ell_j\hs C_j (k)\,,  \label{eq009} \\
&& C_j (k) =\ds\int_L Q_j (\xi)\e^{\hs ik\xi}\, d\xi\,.    \label{eq010}
\eea
%
Here $C_j (k)$ is defined as the Kochin function of each component in the 
disturbance potential and $C(k)$ in Eq.\,(\ref{eq009}) is the total 
Kochin function for the motion-free case. 
The complex amplitude of the $j$-th mode of motion $X_j$ must be 
determined from the coupled motion equations.
\\

In accordance with Eq.\,(\ref{eq009}) for the Kochin function, 
the ship-generated progressive wave $\zeta (x, y)$ can be written 
in the linear theory as the linear superposition of scattering 
wave $\zeta_7 (x, y)$ and radiation waves $\zeta_j (x, y)$ 
by surge ($j=1$), heave ($j=3$), and pitch ($j=5$) motions, 
in the form
\be
\zeta (x, y) = A\hs\zeta_7(x, y) +
\ds\sum_{j=1,3,5} X_j\hs\ell_j\hs \zeta_j (x, y)\,. \label{eq011}
\ee
Noting that the wavenumbers $k_j\ (j=1\sim 4)$ appearing as the limits 
of integration in Eq.\,(\ref{eq005}) are the roots of $\kappa^2 = k^2$ 
and $\epsilon_k ={\rm sgn} (\omega + kU)=-1$ for $ -\infty <k<k_1$ 
and $\epsilon_k=1$ for $k_2 < k < \infty$, the elevation of 
progressive wave, Eq.\,(\ref{eq005}), can be expressed in the form
\be
\zeta (x, y) =\ds\frac{i}{2\pi} \ds\int_L 
u(\kappa^2 -k^2)\, C(k) 
\sqrt{\ds\frac{\kappa}{k_0}}
\ds\frac{\kappa}{\sqrt{\kappa^2 -k^2}}
\e^{-ikx -i\epsilon_k |\hs y\hs|\sqrt{\kappa^2 -k^2}}\, dk\,, \label{eq012}
\ee
where $u (\kappa^2 -k^2)$ is the unit step function, 
equal to 1 for $\kappa^2 > k^2$ and zero otherwise.
\\

Let us consider the Fourier transform of $\zeta (x, y)$ with respect to $x$, 
defined by the following integral:
\be
\zeta^* (\ell, y)=\ds\int_{-\infty}^{\hs \infty} 
\zeta (x, y)\e^{\hs i\ell x}\, dx\,.                  \label{eq013}
\ee
Substituting Eq.\,(\ref{eq012}) in Eq.\,(\ref{eq013}) and using an integral 
representation of Dirac's delta function
\be
\ds\frac{1}{2\pi} \ds\int_{-\infty}^{\hs\infty} 
e^{\hs i(\ell -k)x}\, dx = \delta (\ell -k )\,,        \label{eq014}
\ee
we can obtain with relative ease the following relation:
\be
\zeta^* (k, y) = i\, C(k)\hs 
\sqrt{\frac{\kappa}{k_0}} \frac{\kappa}{\sqrt{\kappa^2 -k^2}}
\e^{-i\epsilon_k |\hs y\hs|\sqrt{\kappa^2 -k^2}} \,.       \label{eq015}
\ee

According to Maruo's theory (1960), the added resistance in head waves 
can be computed in terms of the Kochin function as follows:
\bea
R_{AW} &=& \ds\frac{\rho g}{4\pi k_0} \ds\int_{-\infty}^{\hs\infty} 
\epsilon_k\, u(\kappa^2 -k^2) \,\big|\, C(k)\hs\big|^2 
\ds\frac{\kappa}{\sqrt{\kappa^2 -k^2}}
\hs (k+k_0)\, dk   \nonumber \\
&=& \ds\frac{\rho g}{4\pi k_0}
\bigg[ -\!\int_{-\infty}^{\, k_1} \! +  \!\int_{k_2}^{\, k_3} \! 
+\!\int_{k_4}^{\,\infty} \bigg]\,
\big|\, C(k)\hs\big|^2
\ds\frac{\kappa}{\sqrt{\kappa^2 -k^2}}
\hs (k + k_0 )\, dk\,.                         \label{eq016}
\eea
Therefore, substituting Eq.\,(\ref{eq015}) in Eq.\,(\ref{eq016}) provides a formula 
for computing the added resistance  with the Fourier transform of 
ship-generated unsteady waves, in the form
\be
R_{AW} =\ds\frac{\rho g}{4\pi}\,
\bigg[ -\!\int_{-\infty}^{\, k_1} \! +  \!\int_{k_2}^{\, k_3} \! 
+\!\int_{k_4}^{\,\infty} \bigg]\,
\big|\, \zeta^* (k, y)\hs\big|^2
\ds\frac{\sqrt{\kappa^2 -k^2}}{\kappa^2}\hs 
(k+k_0 )\, dk\,.                                \label{eq017}
\ee

Here we should note a few things regarding 
the wavenumbers $k_j\ (j=1\sim 4)$ appearing in Eq.\,(\ref{eq017}).
First, for $\tau > 1/4$, $k_3$ and $k_4$ become complex as is obvious from 
Eq.\,(\ref{eq008}), and the integration 
range in Eq.\,(\ref{eq017}) must be treated as continuous for $k_2 < k$.
Next, $\omega = \omega_0 + k_0 U$ holds in head waves, which 
gives the following relations:
\be
\left. \!\!\! \begin{array}{l}
\omega_0 = \ds\frac{g}{2U}\big(\hs -1 +\sqrt{1+4\tau}\hs\big) >0 \\[2mm]
k_0=\ds\frac{\omega_0^2}{g} =\frac{K_0}{2}
\big(\hs 1+2\tau -\sqrt{1+4\tau}\hs\big)
=-k_2 =|k_2|
\end{array} \right\}           \label{eq018}
\ee

On the other hand, it can be shown that the relation between the ship's speed $U$ 
and the phase velocity $c$ of component wave $k_j\ (j=1, 3, 4)$ along the $x$-axis 
is given by
\be
\left. \!\!\! \begin{array}{ll}
0 < U < \ds\frac{c}{2} & \mbox{for}\ k_3\mbox{-wave} \\[1mm]
\ds\frac{c}{2} < U < c & \mbox{for}\ k_4\mbox{-wave} \\[1.5mm]
c < U                  & \mbox{for}\ k_1\mbox{-wave} 
\end{array} \right\}               \label{eq019}
\ee
Since $c/2$ is equal to the group velocity with which the energy 
of progressive wave is transported, 
we can understand the location of existence, the relative wavelength, 
and the propagation direction when viewed from a ship moving at forward speed $U$ 
for each of the component waves $k_j\ (j=1\sim 4)$; these are schematically 
shown in Fig.\,\ref{fig001}.
It is noteworthy that at $\tau =1/4$, $k_3 =k_4$ and $U$ becomes equal to 
the group velocity of progressive wave.


\medskip%%%%%%%%%%%%%%%%%%%%%%%%%%%%%%%%%%%%%%%%%%%%%%%%%%%%%%%%%%%%%%
%%%%%%%%%%%%%%%%%%%%%%%%%%% Next Section %%%%%%%%%%%%%%%%%%%%%%%%%%%%%
%%%%%%%% Please use \section*{ .... } or \jsection{ ..... } %%%%%%%%%%
%%%%%%%%%%%%%%%%%%%%%%%%%%%%%%%%%%%%%%%%%%%%%%%%%%%%%%%%%%%%%%%%%%%%%%
\jsection{Dominant Components in Added Resistance}
%---------------------------------------------------------------------
%
In reality, the wavenumber of progressive wave varies 
over the integration range shown in Eq.\,(\ref{eq017}).
In order to see which component of progressive waves contributes predominantly 
to the added resistance, we will evaluate the values of the integrand of 
Eq.\,(\ref{eq017}), by rewriting Eq.\,(\ref{eq017}) in the form
\be
R_{AW} =
\ds\frac{\rho\hs g}{4\pi}\,
\bigg[ -\!\int_{-\infty}^{\, k_1} \! +  \!\int_{k_2}^{\, k_3} \! 
+\!\int_{k_4}^{\,\infty} \bigg]\,
W_1 (k)\,W_2 (k)\, dk\,,           \label{eq020}
\ee
where
\bea
&& W_1 (k) =\big|\, \zeta^* (k, y)\hs\big|^2\,,     \label{eq021} \\
&& W_2 (k) = \ds\frac{\sqrt{\kappa^2 -k^2}}{\kappa^2}\, 
\big(\, k+k_0\hs\big)\,.                          \label{eq022}
\eea
%
We should note that $W_2(k)=0$ at $k=k_j\ (j=1\sim 4)$ because of $\kappa^2 =k^2$, 
and $k+k_0 = k-k_2$ by Eq.\,(\ref{eq018}).
When computing the added resistance for the case of forced oscillation 
problem (the so-called radiation problem), $k+k_0$ in Eq.\,(\ref{eq022}) must be replaced 
simply with $k$, because the term related to $k_0$ in Eq.\,(\ref{eq022}) represents 
interactions between the incident wave and ship-generated disturbance waves.
\\

In order to understand qualitatively the dominant wave components and 
general characteristics in the Fourier transform of progressive waves, 
it may be informative to 
consider a simplified wave profile.
For that purpose, let us consider a wave component, 
propagating in the positive $x$-axis with 
wavenumber $k_\ell$ and amplitude of the following form:
\be
\zeta (x, y) =\alpha\, 
\ds\frac{u(x_s -x)}{\sqrt{\hs |x-x_s|}}
\e^{-i k_\ell x}\,,             \label{eq023}
\ee
where $\alpha$ denotes the amplitude coefficient, $x_s$ the starting point 
of wave existence along a line parallel to the $x$-axis, and
$u(x_s -x)$ the unit step function.
\\

The Fourier transform of this wave may be expressed as
\bea
\zeta^* (k, y) &=& \ds\int_{-\infty}^{\hs\infty}
 \zeta (x, y)\e^{\hs ikx}\,dx \nonumber \\
&=& \alpha \sqrt{\ds\frac{\pi}{|k - k_\ell |}}
\e^{\hs i (k-k_\ell)x_s}
\e^{\hs i\frac{\pi}{4} {\rm sgn}(k- k_\ell)}\,.  \label{eq024}
\eea
Therefore it is obvious that the value of $W_1 (k)$ becomes large 
at $k=k_\ell$ and decays in proportion to $1/|k-k_\ell |$.
\\

For larger values of $k$, $W_2 (k)$ becomes small with order of $O(1/k)$ and
the amplitude coefficient $\alpha$ must be small in reality 
for waves with large $k$ (short wavelength).
As already noted, $k_j\ (j=1\sim 4)$ is a root of $\kappa^2 -k^2=0$, 
and $k+k_0 = k-k_2$.
Hence, dominant wave components in the added resistance may be 
relatively longer waves with smaller value of $k$ satisfying $k_2 < k$.
We note that if $k_2 < k <0$, the wave propagates in the 
negative $x$-axis like $k_2$-wave in Fig.\,\ref{fig001}, and 
if $0< k$, the wave propagates in the positive $x$-axis 
like $k_3$- and $k_4$-waves in Fig.\,\ref{fig001}.



\medskip%%%%%%%%%%%%%%%%%%%%%%%%%%%%%%%%%%%%%%%%%%%%%%%%%%%%%%%%%%%%%%
%%%%%%%%%%%%%%%%%%%%%%%%%%% Next Section %%%%%%%%%%%%%%%%%%%%%%%%%%%%%
%%%%%%%% Please use \section*{ .... } or \jsection{ ..... } %%%%%%%%%%
%%%%%%%%%%%%%%%%%%%%%%%%%%%%%%%%%%%%%%%%%%%%%%%%%%%%%%%%%%%%%%%%%%%%%%
\jsection{Experiments}
%---------------------------------------------------------------------
%
Experiments were carried out in head waves, measuring the added resistance by 
a dynamometer, ship-generated unsteady waves using a larger number of wave probes 
of capacitance type, and also wave-induced ship motions in the motion-free case.
In the measurement of unsteady waves, 6 wave probes were used in the first 
experiment, but the number of wave probes was increased up to 12 in the 
second experiment to confirm the resolution accuracy particularly for 
short-wavelength waves.
%
Those wave probes were fixed in space and positioned with almost equal 
interval over the distance of ship's movement in one period of encounter 
along a longitudinal line parallel to the $x$-axis (at constant $y$).
\\

The ship-generated wave is expressed as the Fourier-series expansion 
of the form
\bea
\zeta_W (x, y, t) &=& \zeta_0 (x, y) +
\zeta_c (x, y) \cos \omega t + \zeta_s (x, y) \sin \omega t \nonumber \\
&=& \zeta_0 (x, y) +{\rm Re}\hs \Big[\hs 
\big\{ \zeta_c (x, y) -i\,\zeta_s (x, y)\big\}
\e^{\hs\iomega t}\,\Big]\,.                      \label{eq025}
\eea
The spatial distributions of time-independent steady component, $\zeta_0 (x, y)$, 
and unsteady cosine and sine components, $\zeta_c (x, y)$ and $\zeta_s (x, y)$, 
in Eq.\,(\ref{eq025}) were obtained by the least-squares method using 
the data measured with 12 wave probes.
%
The Fourier transform of the measured unsteady wave, denoted as $\zeta (x, y)
=\zeta_c (x, y) -i\,\zeta_s (x, y)$, was computed as follows.
Suppose that the range of $x$ in actual measurement is from $b$ to $a$ 
and the number of total data points is $N+1$.
Then by assuming linear variation between adjacent data points 
$(\,\zeta_n \sim \zeta_{n+1}\,)$, we integrate analytically with 
respect to $x$ over each segment of data points.
The result of this analytical integration can be expressed as
\be
\zeta^* (k, y) \simeq \ds\int_b^{\, a}
\zeta (x, y)\e^{\hs ikx}\, dx
=\ds\sum_{n=2}^N \gGamma_n (k)\, \zeta_n\,,     \label{eq026}
\ee
where 
\bea
\gGamma_n (k) = &&\,\left[ \ds\frac{e^{\hs ikx}}
{ik (x_n - x_{n-1})} \bigg\{ x-x_{n-1}- \ds\frac{1}{ik}
\bigg\}\hs\right]_{x=x_{n-1}}^{x=x_n} \nonumber \\
&&\, + \left[ \ds\frac{e^{\hs ikx}}
{ik (x_n - x_{n+1})} \bigg\{ x-x_{n+1}- \ds\frac{1}{ik}
\bigg\}\hs\right]_{x=x_n}^{x=x_{n+1}}        \label{eq027}
\eea

The ship model used in the experiments is 
a modified Wigley model with wider breadth, 
expressed mathematically as
\be
\left. \!\!\! \begin{array}{l}
\eta = (1-\zeta^2)(1-\xi^2)(1+0.6\xi^2+\xi^4)
 +\zeta^2 (1-\zeta^8)(1-\xi^2)^4   \\[1.5mm]
\xi = x\ds\frac{2}{L},\ \ 
\zeta =y\ds\frac{2}{B}, \ \ \zeta =\ds\frac{z}{d}
\end{array} \right\}                      \label{eq028}
\ee
where the real dimensions are $L=$ 2.5 m, $B=$ 0.5 m, $d=$ 0.175 m.
The gyrational radius in pitch and the center of gravity were 
set equal to $\kappa_{yy}/L=0.236$ and $\ubar{OG} /d =0.177$ (below 
the free surface) in the first experiment and 
$\kappa_{yy}/L=0.238$ and $\ubar{OG} /d =0.189$ in the second
experiment; the influence of the difference in these values 
on obtained results concerned with the present study may be negligible.
\\

The lateral distance of a longitudinal line used for the wave 
measurement from the centerline of a ship ($x$-axis) was set equal 
to $y=B/2+$ 0.1 m $=$ 0.35 m.
The Froude number was $Fn=0.2$ in all measurements.
\\

In the second experiment, to see the degree of contribution of 
each component wave $\zeta_j (x, y)$ defined in Eq.\,(\ref{eq011}) 
in the linear superposition 
of ship-generated unsteady waves to the added resistance, 
the experiments were conducted for the cases of wave 
diffraction (where ship motions are fixed), forced oscillation 
in heave and pitch (where incident waves are absent), and 
free oscillation in waves (where surge, heave, and pitch are 
free to respond to waves).


\medskip%%%%%%%%%%%%%%%%%%%%%%%%%%%%%%%%%%%%%%%%%%%%%%%%%%%%%%%%%%%%%%
%%%%%%%%%%%%%%%%%%%%%%%%%%% Next Section %%%%%%%%%%%%%%%%%%%%%%%%%%%%%
%%%%%%%% Please use \section*{ .... } or \jsection{ ..... } %%%%%%%%%%
%%%%%%%%%%%%%%%%%%%%%%%%%%%%%%%%%%%%%%%%%%%%%%%%%%%%%%%%%%%%%%%%%%%%%%
\jsection{Results and Discussion}
%---------------------------------------------------------------------
\vspace{-0.2mm}\par
\noindent
Figures \ref{fig02} through \ref{fig05} show comparisons of the added 
resistance for the cases of wave diffraction (Fig.\,\ref{fig02}), 
forced heave (Fig.\,\ref{fig03}), forced pitch (Fig.\,\ref{fig04}), 
and free oscillation in waves (Fig.\,\ref{fig05}).
%
Basically the results measured directly by a dynamometer are shown by 
closed circles, the results obtained from the unsteady wave analysis are 
shown by open circles, and computed results by Enhanced Unified Theory (EUT) 
are shown by the solid line or other symbols.
It is confirmed in numerical computations by EUT that the added resistances 
computed directly from the Kochin function and from the wave 
analysis method using computed wave profile are virtually the same.
\\

In Fig.\,\ref{fig02}, we can observe that the results by the unsteady wave analysis 
are in good agreement with computed ones by EUT, but those are almost 
half of the value by the direct measurement irrespective of 
the wavelength tested.
This implies that nonlinear local wave generation, including wave breaking, 
may exist in the wave diffraction problem.
\\

In the forced-oscillation problem shown as Figs.\,\ref{fig03} and \ref{fig04}, 
the agreement between EUT and the results of unsteady wave analysis is 
generally favorable, and noticeable discrepancy from the results by the 
direct measurement can be observed in the short-wavelength range 
especially in forced pitch oscillation.
This discrepancy may be attributed to nonlinear local wave generation 
which cannot be explained by a linear potential-flow theory. 
However in this short-wavelength range, actual amplitudes of ship motions 
are normally very small. Thus little effects will arise from this 
discrepancy on the total value of the added resistance.
\\

%---------------------------------------------------------------------
\begin{figure}[b]% 0 0 591 246
\bc
\includegraphics[width=0.9\linewidth]{./figure/Raw-Diffraction.eps}
\ec
\par\vspace*{-7mm}
\caption{Added resistance in the diffraction problem
on modified Wigley model at $Fn=0.2$} \label{fig02}
%\end{figure}
%---------------------------------------------------------------------
%---------------------------------------------------------------------
%\begin{figure}[h]% 0 0 591 246
\bc
\includegraphics[width=0.9\linewidth]{./figure/Raw-Heave.eps}
\ec
\par\vspace*{-7mm}
\caption{Added resistance in forced heave oscillation 
($X_3=$ 0.01 m) on modified Wigley model at $Fn=0.2$} \label{fig03}
%\end{figure}
%---------------------------------------------------------------------
%---------------------------------------------------------------------
%\begin{figure}[h]% 0 0 591 246
\bc
\includegraphics[width=0.9\linewidth]{./figure/Raw-Pitch.eps}
\ec
\par\vspace*{-7mm}
\caption{Added resistance in forced pitch oscillation 
($X_5=$ 1.4 deg) on modified Wigley model at $Fn=0.2$} \label{fig04}
\end{figure}
%---------------------------------------------------------------------

%---------------------------------------------------------------------
\begin{figure}[tbh]% 0 0 591 246
\bc
\includegraphics[width=0.9\linewidth]{./figure/Raw-MotionFree.eps}
\ec
\par\vspace*{-7mm}
\caption{Added resistance in waves (motion free) 
on modified Wigley model at $Fn=0.2$} \label{fig05}
\par\vspace*{-1mm}
\end{figure}
%---------------------------------------------------------------------

%---------------------------------------------------------------------
\begin{figure}[thb]% 0 0 591 246
\bc
\includegraphics[width=0.97\linewidth]{./figure/Waves-All-110.eps}
\ec
\par\vspace*{-4mm}
\caption{Wave profiles generated by modified Wigley model at $Fn=0.2$.
(a): Scattering wave in the diffraction problem at $\lambda /L=1.1$, 
(b): Radiation wave by forced heave oscillation at $KL=12.5$ and $X_3=$ 0.01 m,
(c): Radiation wave by forced pitch oscillation at $KL=12.5$ and $X_5=$ 1.36 deg,
(d): Superimposed wave using the waves of (a)$\sim$(c) and measured complex 
amplitudes of heave and pitch at $\lambda /L=1.1$,
(e): Measured wave in the motion-free condition at $\lambda /L=1.1$.} \label{fig06}

\end{figure}
%---------------------------------------------------------------------


%---------------------------------------------------------------------
\begin{figure}[thb]% 0 0 591 246
\bc
\includegraphics[width=0.97\linewidth]{./figure/Waves-All-095.eps}
\ec
\par\vspace*{-4mm}
\caption{Wave profiles generated by modified Wigley model at $Fn=0.2$.
(a): Scattering wave in the diffraction problem at $\lambda /L=0.95$, 
(b): Radiation wave by forced heave oscillation at $KL=15.0$ and $X_3=$ 0.01 m,
(c): Radiation wave by forced pitch oscillation at $KL=15.0$ and $X_5=$ 1.36 deg,
(d): Superimposed wave using the waves of (a)$\sim$(c) and measured complex 
amplitudes of heave and pitch at $\lambda /L=0.95$,
(e): Measured wave in the motion-free condition at $\lambda /L=0.95$.} \label{fig07}

\end{figure}
%---------------------------------------------------------------------


%---------------------------------------------------------------------
\begin{figure}[thb]% 0 0 591 246
\bc
\includegraphics[width=0.97\linewidth]{./figure/Waves-All-0517.eps}
\ec
\par\vspace*{-4mm}
\caption{Wave profiles generated by modified Wigley model at $Fn=0.2$.
(a): Scattering wave in the diffraction problem at $\lambda /L=0.517$, 
(b): Radiation wave by forced heave oscillation at $KL=35.0$ and $X_3=$ 0.01 m,
(c): Radiation wave by forced pitch oscillation at $KL=35.0$ and $X_5=$ 1.36 deg,
(d): Superimposed wave using the waves of (a)$\sim$(c) and measured complex 
amplitudes of heave and pitch at $\lambda /L=0.517$,
(e): Measured wave in the motion-free condition at $\lambda /L=0.517$.} \label{fig08}

\end{figure}
%---------------------------------------------------------------------



Measured results shown in Fig.\,\ref{fig05} for the motion-free case are 
the results obtained in the first experiment, but these are 
essentially the same as those obtained in the second experiment.
A large discrepancy can be seen between the results by the direct 
measurement and the wave analysis using measured waves, particularly near 
the peak around $\lambda /L =1.1$.
\\

In order to investigate a possible reason for this discrepancy, 
the wave profile was computed by the linear superposition 
according to Eq.\,(\ref{eq011}), using the component waves 
obtained by the experiments of wave diffraction ($j=7$), 
forced heave ($j=3$), and forced pitch ($j=5$), 
together with complex amplitudes of heave and pitch motions 
measured in the motion-free experiment.
(The surge mode is ignored, because the forced oscillation test 
in surge could not be conducted.)
%
Then the superimposed wave profile was Fourier-transformed and the 
added resistance was computed from Eq.\,(\ref{eq017}).
The results of added resistance obtained from this linear superposition 
of component waves and complex motion amplitudes are also 
shown in Fig.\,\ref{fig05} with diamond symbol.
It is remarkable that 
these results become much closer to the results by the direct measurement 
and computed by EUT, especially near the peak where ship motions also 
become large.
\\

Figure\,\ref{fig06} provides the information on the profiles of scattering and
radiation waves and on the difference between the wave profiles 
obtained by the linear superposition without surge motion and 
measured by the motion-free experiment, 
for a case of $\lambda /L =1.1$ (which corresponds 
approximately to $KL=12.5$ in the radiation problem at $Fn=0.2$).
\\

From these figures, we can see that 
the overall appearance of wave profile is very similar between 
superimposed and directly measured waves, but a large difference exists 
near the fore-front part of the wave.
The source of this difference seems to come from the wave by the forced 
pitch oscillation. It is noteworthy that the forced oscillation tests were 
performed with relatively small amplitude ($X_3 =$ 0.01 m and $X_5=$ 1.36 deg) 
within the range of linear theory being valid.
Therefore, when the amplitude of ship motions becomes large,
linearity in the amplitude of generated wave 
may be violated particularly near the ship's bow due to 
large pitch motion, and as a result, some nonlinear local waves 
with energy dissipation may be generated.
\\

In order to provide similar information at different wavelengths on the degree 
of difference between the wave profiles measured in the motion-free 
experiment and obtained by the linear superposition of component waves, 
the result for $\lambda /L =0.95$ (which corresponds to $KL=15.0$ in 
the radiation problem at $Fn=0.2$) 
is shown in Fig.\,\ref{fig07}, and the result for $\lambda /L =0.517$ 
(which corresponds to $KL=35.0$ in the radiation problem 
at $Fn=0.2$) is shown in Fig.\,\ref{fig08}.
\\

In Fig.\,\ref{fig07}, a clear difference can also be seen between 
(d) and (e) near the fore-front part of the wave, although the magnitude 
in the difference is smaller than that for the case of 
$\lambda /L =1.1$ in Fig.\,\ref{fig06}.
This difference explains the fact that the added resistance obtained 
from the measured wave in the motion-free case is smaller than the 
value obtained from the superimposed wave and also smaller than the value by the 
direct measurement with dynamometer.
\\

In Fig.\,\ref{fig08}, we can see that the difference between 
superimposed wave (d) and directly measured wave (e) 
is very small and the wave profile in (d) or (e) is very similar 
to the scattering wave shown as (a).
This is because the ship motions at $\lambda /L=0.517$ are 
relatively small and their contributions to the superimposed wave 
are negligible.
As a result, the values of added resistance from the superimposed 
and directly measured waves are almost the same.
However, those values are still approximately half of the value 
by the direct measurement with dynamometer, as shown in Fig.\,\ref{fig05}.


\medskip%%%%%%%%%%%%%%%%%%%%%%%%%%%%%%%%%%%%%%%%%%%%%%%%%%%%%%%%%%%%%%
%%%%%%%%%%%%%%%%%%%%%%%%%%% Next Section %%%%%%%%%%%%%%%%%%%%%%%%%%%%%
%%%%%%%% Please use \section*{ .... } or \jsection{ ..... } %%%%%%%%%%
%%%%%%%%%%%%%%%%%%%%%%%%%%%%%%%%%%%%%%%%%%%%%%%%%%%%%%%%%%%%%%%%%%%%%%
\jsection{Conclusions}
%---------------------------------------------------------------------
%
By using the unsteady waves measured in the diffraction 
and radiation (heave and pitch only) problems and the complex amplitude 
of wave-induced motions, the unsteady wave corresponding to the one in  
the motion-free condition was produced by the linear superposition.
The overall agreement in the wave profile was favorable, but a prominent difference 
was observed in the fore-front part of the wave, especially when the 
ship motions are large. The added resistance computed from this superimposed 
wave was in better agreement with the directly measured value.
We can envisage from these results that, when ship motions become large, some 
nonlinear local waves may be generated, resulting in a noticeable discrepancy 
in the added resistance between the results of direct measurement and unsteady 
wave analysis.

\newpage\smallskip
\medskip%%%%%%%%%%%%%%%%%%%%%%%%%%%%%%%%%%%%%%%%%%%%%%%%%%%%%%%%%%%%%%
%%%%%%%%%%%%%%%%%%%%%%%%%%% Next Section %%%%%%%%%%%%%%%%%%%%%%%%%%%%%
%%%%%%%% Please use \section*{ .... } or \jsection{ ..... } %%%%%%%%%%
%%%%%%%%%%%%%%%%%%%%%%%%%%%%%%%%%%%%%%%%%%%%%%%%%%%%%%%%%%%%%%%%%%%%%%
\jsection{Acknowledgments}
%---------------------------------------------------------------------
%
The experiments for the unsteady wave analysis in the present study 
have been carried out at the towing tank of 
Research Institute for Applied Mechanics (RIAM), Kyushu University, as a joint 
experiment with Hiroshima University.
The authors would like to thank Prof. Changhong Hu of Kyushu University for 
his permission of using the towing tank of RIAM and Prof. Hidetsugu Iwashita of 
Hiroshima University for his great help in the measurement and 
subsequent analyses for unsteady waves.

%----------------------------------------------------------------------------
%\lastpagecontrol{50mm} %% This size must be adjusted depending on your paper
%----------------------------------------------------------------------------

\medskip%%%%%%%%%%%%%%%%%%%%%%%%%%%%%%%%%%%%%%%%%%%%%%%%%%%%%%%%%%%%%%
%%%%%%%%%%%%%%%%%%%%%%%%%%%% REFERENCES %%%%%%%%%%%%%%%%%%%%%%%%%%%%%%
%%%%%%%% Please use \jsection{ .... } or \section*{ ..... } %%%%%%%%%%
%%%%%%%%%%%%%%%%%%%%%%%%%%%%%%%%%%%%%%%%%%%%%%%%%%%%%%%%%%%%%%%%%%%%%%
\section*{REFERENCES} \par\vspace*{-0.0mm}

\hspace*{4mm}\begin{minipage}[t]{87.0mm}
\noindent\hspace*{-5.0mm}
Kashiwagi, M (1995). \lq\lq Prediction \hs of \hs Surge and Its Effect on 
Added Resistance by Means of the Enhanced Unified Theory\rq\rq, 
{\it Trans West-Japan Society of Naval Architects}, No\,\,89, pp\,77\hs--89.
\end{minipage}\hfill

\vspace*{1.5mm}

\hspace*{4mm}\begin{minipage}[t]{87.0mm}
\noindent\hspace*{-5.0mm}
Kashiwagi, M, Ikeda, T and Sasakawa, T (2010). 
\lq\lq Effects of Forward Speed of a Ship on Added Resistance in Waves\rq\rq, 
{\it Intl J Offshore \& Polar Engineering}, Vol\,\,20, No\,\,3, pp\,196\hs--203.
\end{minipage}\hfill

\vspace*{1.5mm}

\hspace*{4mm}\begin{minipage}[t]{87.0mm}
\noindent\hspace*{-5.0mm}
Kashiwagi, M (2010). 
\lq\lq Prediction of Added Resistance by Means of Unsteady Wave-Pattern Analysis\rq\rq, 
{\it Proc 25th Intl Workshop on Water Waves and Floating Bodies}, Harbin, 
pp\,69\hs--72.
\end{minipage}\hfill

\vspace*{1.5mm}

\hspace*{4mm}\begin{minipage}[t]{87.0mm}
\noindent\hspace*{-4.8mm}
Maruo, H (1960). \lq\lq Wave resistance of a ship in regular head seas\rq\rq, 
{\it Bulletin of the Faculty of Engineeering, Yokohama National University}, 
Vol\,\,9, pp\,73\hs--91.
\end{minipage}\hfill

\vspace*{1.5mm}

\hspace*{4mm}\begin{minipage}[t]{87.0mm}
\noindent\hspace*{-4.8mm}
Ohkusu, M (1980).  \lq\lq
Added Resistance in Waves in the Light of Unsteady Wave 
Pattern Analysis\rq\rq, {\it Proc of 13th Symposium on Naval Hydrodynamics}, Tokyo, 
pp\,413--425.
\end{minipage}\hfill

\vspace*{1.5mm}

\hspace*{4mm}\begin{minipage}[t]{87.0mm}
\noindent\hspace*{-4.75mm}
Wakabayashi, T, Sasakawa, T and Kashiwagi, M (2010). 
\lq\lq Study on Added Resistance with Unsteady Wave-Pattern Analysis \rq\rq, 
{\it Proc 5th Asia-Pacific Workshop on Marine Hydrodynamics - APHydro 2010}, Osaka, 
pp\,1\hs--4.
\end{minipage}\hfill
 
%----------------------------------------------------------------------
%\lastpagesettings
%%%%%%%%%%%%%%%%%%%%%%%%%%%%%%%%%%%%%%%%%%%%%%%%%%%%%%%%%%%%%%%%%%%%%%%
%

\end{document}
