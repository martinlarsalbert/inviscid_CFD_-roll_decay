\section{Results}\label{results}

    \subsection{Roll decay model tests}\label{roll-decay-model-tests}

\subsubsection{0 knots}\label{knots}

Data from two roll decay model tests conducted at zero speed were
available:
\href{../../notebooks/01.2_select_suitable_MDL_test_KLVCC2.ipynb\#rolldecay}{Roll
decay model tests 0 knots}. These tests where analyzed by fitting a
\href{../../notebooks/01.2_select_suitable_MDL_test_KLVCC2.ipynb\#cubic_model}{Cubic
model} to the model test data. The two models were very similar in terms
of roll damping and stiffness, suggesting good repeatability in the
model tests as well as in the parameter identification technique (PIT)
used. The individual damping from each oscillation obtained with the
logarithmic decrement method are very scattered, but this does not seem
to influence the two models for the 0 speed case, which are very
similiar.

    \subsubsection{15.5 knots}\label{knots}

Data from one roll decay model tests conducted at a ship speed
corresponding to 15.5 knots full scale ship speed was also available.
This model tests was analyzed in the same way as the other tests. It was
found that the damping was higher at speed. The ship got a small
\href{../../notebooks/01.3_select_suitable_MDL_test_KLVCC2_speed.ipynb\#yawrate}{yaw
rate} at the end of test, giving a small steady roll angle due to the
centrifugal force. Since this effect is not included in the matematical
model used, the steady roll angle was instead removed by removing the
linear trend in the roll angle signal.

    \begin{figure}
        \begin{center}\adjustimage{max size={0.9\linewidth}{0.4\paperheight}}{figures/mdl.pdf}\end{center}
        \caption{Model test results}
        \label{fig:mdl}
    \end{figure}
    
    \subsection{Ikeda's method}\label{ikedas-method}

    When looking at predictions for KVLCC2 at 0 speed made with regular
Ikeda's method, it was found that the eddy damping \(B_E\) was too high
compared with the regular implementation, compared to the model test
results. Eventhough the rest of the components would also be
overpredicted, the \(B_E\) would still be too large. The eddy damping
calculated with \(C_r\) predicted with the descision tree gave much
better agreement.

    \begin{figure}
        \begin{center}\adjustimage{max size={0.9\linewidth}{0.4\paperheight}}{figures/ikeda.pdf}\end{center}
        \caption{Ikeda prediction}
        \label{fig:ikeda}
    \end{figure}
    
    \subsection{FNPF}\label{fnpf}

Simulations of roll decay tests were conducted with FNPF for each model
test. The results from these simulations can be seen in the figure
below.

    \begin{figure}
        \begin{center}\adjustimage{max size={0.9\linewidth}{0.4\paperheight}}{figures/fnpf.pdf}\end{center}
        \caption{FNPF}
        \label{fig:fnpf}
    \end{figure}
    
    \subsection{Roll damping prediction with Hybrid
method}\label{roll-damping-prediction-with-hybrid-method}
Replacing the wave damping $B_W$, for the zero speed case above, with values obtained with FNPF is shown below. 
    \begin{figure}
        \begin{center}\adjustimage{max size={0.9\linewidth}{0.4\paperheight}}{figures/hybrid_0.pdf}\end{center}
        \caption{Hybrid method 0 knots}
        \label{fig:hybrid_0}
    \end{figure}
    
    FNPF gave stable results at zero speed, but very unstable results at
speed, due to what was thought to be memory effects. For the speed case
the viscous damping was instead present during the actual FNPF
simulations. The viscous damping was included by adding a linear
\(B_{visc,1}\) and quadratic \(B_{visc,2}\) damping term the equation of
motions in the FNPF. The results are compared with corresponding results
from the model tests.

    \begin{figure}
        \begin{center}\adjustimage{max size={0.9\linewidth}{0.4\paperheight}}{figures/hybrid_speed_amplitudes.pdf}\end{center}
        \caption{Hybrid method at speed amplitudes}
        \label{fig:hybrid_speed_amplitudes}
    \end{figure}
    
    The results using the Hybrid method for Run 3 at speed gives very
similar results to the corresponding model tests.

    \begin{figure}
        \begin{center}\adjustimage{max size={0.9\linewidth}{0.4\paperheight}}{figures/hybrid_speed_time.pdf}\end{center}
        \caption{Hybrid method at speed}
        \label{fig:hybrid_speed_time}
    \end{figure}
    
    The coefficients obtained from model tests, FNPF and Hybrid method are
summarized in model scale units in the table below:
 
            
    
    \begin{longtable}[c]{@{}lllllll@{}}
\toprule\addlinespace
run & method & $F_n$ & $\omega_0$ & $B_1$ & $B_2$ & $B_3$\\\addlinespace 
\midrule\endhead
1 & model test & 0.0 & 2.461405663797153 & 2.960431148229692 & -6.520487128705247 & 43.7753590783464\\\addlinespace 
2 & model test & 0.0 & 2.461149414341689 & 2.879736279200558 & -5.874151219904037 & 41.50366418695269\\\addlinespace 
2 & FNPF & 0.0 & 2.469512621237472 & 0.14245882294522433 & 2.9442349635220273 & 0.0\\\addlinespace 
3 & model test & 0.14231828016230666 & 2.4675051791344322 & 7.523313711481077 & 7.0280627607638 & 0.3897887036233833\\\addlinespace 
3 & FNPF & 0.14231828016230666 & 2.4731115844417735 & 1.939382895537425 & 2.876088787299491 & 0.0\\\addlinespace 
3 & FNPF & 0.14231828016230666 & 2.4672709863288573 & 1.8288702419438334 & 3.3099823707085987 & 0.0\\\addlinespace 
3 & FNPF & 0.14231828016230666 & 2.4531795386475976 & 2.503610883554999 & 1.3308576126192526 & 0.0\\\addlinespace 
3 & FNPF & 0.14231828016230666 & 2.4642154500169404 & 7.612597393349741 & 8.176912123526904 & 0.0\\\addlinespace 
3 & FNPF & 0.14231828016230666 & 2.440717566554755 & 7.8985005775656605 & 4.593428224531338 & 0.0\\\addlinespace 
\bottomrule 
 \end{longtable}

    

    