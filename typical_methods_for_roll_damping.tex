\section{Typical methods for roll
damping}\label{typical-methods-for-roll-damping}

    \subsection{Roll decay test}\label{roll-decay-test}

The most common way to determine the roll damping of a ship is to
conduct a roll decay test. The initial heel angle during this test gives
the ship potential energy that is shifting to kinetic energy as the ship
starts to move during the inital phase of the roll decay test. The
energy is transfered between kinetic and potential energy during the
oscillations. The ship loses energy over time due to the damping wich
can be seen in this
\href{../../notebooks/02.2_ikeda_Be_assumption.ipynb\#energy}{plot}
\includegraphics{../figures/energy_transfer.png} Roll decay tests in
model scale experiments as well as from FNPF simulations are used in
this paper to determine the roll damping of KVLCC2. Two different
techniques to identidy the damping from these tests are used: the PIT
method and the logartithmic decrement method which are both described
below.

\subsection{Ikeda's method}\label{ikedas-method}

As a cheaper option to do model tests Ikeda's method can be used to
predict the roll damping. In this method the damping is divided in
various components. This feature enables the possiblity to combine
Ikeda's method with a potential flow method. Ikeda's method is combined
with FNPF giving the Hybrid method as proposed in this paper.

    \subsection{PIT method}\label{pit-method}

    A parameters identification technique (PIT) similar to
\cite{7505983/EXYJELCU} is used to obtain the damping coefficients from
the roll decay model test as well as from the FNPF simulated roll decay
tests. In this technique, parameters in a mathematical model are
determined to get the best fit to a roll decay time signal. A derivation
of a matematical model suitable for this study is described below
together with a description of how the parameters: damping, stiffness
and inertia coefficients are determined.

A differential equation for a linearly decaying motion can be written
as:
 
            
    
    \begin{equation}
\omega_{0}^{2} y + 2 \omega_{0} \zeta \dot{y} + \ddot{y} = 0
\label{eq:equation}
\end{equation}

    

    Which has an analytical solution \cite{undefined}:
 
            
    
    \begin{equation}
\phi = \left(\left(\frac{\zeta \phi_{0}}{\sqrt{1 - \zeta^{2}}} + \frac{\dot{\phi}_{0}}{\omega_{0} \sqrt{1 - \zeta^{2}}}\right) \operatorname{sin}\left(\omega_{0} t \sqrt{1 - \zeta^{2}}\right) + \phi_{0} \operatorname{cos}\left(\omega_{0} t \sqrt{1 - \zeta^{2}}\right)\right) e^{- \omega_{0} t \zeta}
\label{eq:equation}
\end{equation}

    

    In the usual case of having no initial roll velocity (\(\phi_0=0\)) this
equation can be simlified to:
 
            
    
    \begin{equation}
\phi = \frac{\left(\zeta \operatorname{sin}\left(\omega_{0} t \sqrt{1 - \zeta^{2}}\right) + \sqrt{1 - \zeta^{2}} \operatorname{cos}\left(\omega_{0} t \sqrt{1 - \zeta^{2}}\right)\right) \phi_{0} e^{- \omega_{0} t \zeta}}{\sqrt{1 - \zeta^{2}}}
\label{eq:equation}
\end{equation}

    

    And the damping coefficient \(\zeta\) is very small for ships so that
\(\sqrt{1-\zeta}\) is almost 1 and the solution can be further
simplified, into something that can easily be recognized as a decaying
oscillation:
 
            
    
    \begin{equation}
\phi = \phi_{0} e^{- \omega_{0} t \zeta} \operatorname{cos}\left(\omega_{0} t\right)
\label{eq:equation}
\end{equation}

    

    The equations derived above are the linear special case of a more
generall expression that can be expressed in general form according to
\cite{7505983/FB64RGPF}:
 
            
    
    \begin{equation}
A_{44} \ddot{\phi} + \operatorname{B_{44}}\left(\dot{\phi}\right) + \operatorname{C_{44}}\left(\phi\right) = 0
\label{eq:equation}
\end{equation}

    

    Where \(B_{44}(\dot{\phi})\) and \(C_{44}(\phi)\) are the damping and
stiffness models. A cubic model can be obtained by using cubic damping
and stiffness models:
 
            
    
    \begin{equation}
\operatorname{B_{44}}\left(\dot{\phi}\right) = B_{1} \dot{\phi} + B_{2} \left|{\dot{\phi}}\right| \dot{\phi} + B_{3} \dot{\phi}^{3}
\label{eq:equation}
\end{equation}

    
 
            
    
    \begin{equation}
\operatorname{C_{44}}\left(\phi\right) = C_{1} \phi + C_{3} \phi^{3} + C_{5} \phi^{5}
\label{eq:equation}
\end{equation}

    

    The total equation is then written:
 
            
    
    \begin{equation}
A_{44} \ddot{\phi} + \left(B_{1} + B_{2} \left|{\dot{\phi}}\right| + B_{3} \dot{\phi}^{2}\right) \dot{\phi} + \left(C_{1} + C_{3} \phi^{2} + C_{5} \phi^{4}\right) \phi = 0
\label{eq:equation}
\end{equation}

    

    This mathematical model can be reduced to a quadratic damping model when
\(B_3=0\) and a linear model when \(B_2=B_3=0\). This equation does not
have one unique solutionm however. If all parameters would be multiplied
by a factor \(k\) these parameters would also yield as a solution to the
equation. All parameters are therefore divided by the total added mass
parameters \(A_{44}\), replacing the parameters with new parameters such
as:
 
            
    
    \begin{equation}
B_{1A} = \frac{B_{1}}{A_{44}}
\label{eq:equation}
\end{equation}

    

    The equation is now rewritten with these new parameters which have
unique solutions:
 
            
    
    \begin{equation}
\left(B_{1A} + B_{2A} \left|{\dot{\phi}}\right| + B_{3A} \dot{\phi}^{2}\right) \dot{\phi} + \left(C_{1A} + C_{3A} \phi^{2} + C_{5A} \phi^{4}\right) \phi + \ddot{\phi} = 0
\label{eq:equation}
\end{equation}

    

    The parameters of this equation can be identified using least square fit
if the time signals \(\phi(t)\), \(\dot{\phi}(t)\) and
\(\ddot{\phi}(t)\) are all known. This is the case for the results from
the FNPF simulations but not from the model tests, where only the roll
signal \(\phi(t)\) is known. The other time derivatives can be estimated
using numerical derivation of a low pass filtered roll signal or Kalman
filtered roll signal. The filtering will however introduce some errors
in itself. So instead of using this "Derivation approach", it has been
found that solving the differential equation numerically for guessed
parameter values determined using optimization similarly to what was
used by \cite{7505983/FJHQJJUH} and \cite{7505983/9B7QMVJJ} gives the
best parameter estimation. One problem with this "Integration approach"
is that in order to converge, the optimization needs a resonable first
guess of the parameters. The Derivation approach has therefore been used
as a pre-step to obtain a very good first guess of the parameters that
can be passed on to the Integration approach. This has been used for
both signals from FNPF and model tests where in the latter case
numerical derivation is used.

The differential equation is numerically solved as an intial value
problem, where the initial states for \(\phi(t)\), \(\dot{\phi}(t)\) and
\(\ddot{\phi}(t)\) are used to estimate the following states, by
conducting very small time steps using the follownig expression for the
acceleration:
 
            
    
    \begin{equation}
\ddot{\phi} = - B_{1A} \dot{\phi} - B_{2A} \left|{\dot{\phi}}\right| \dot{\phi} - B_{3A} \dot{\phi}^{3} - C_{1A} \phi - C_{3A} \phi^{3} - C_{5A} \phi^{5}
\label{eq:equation}
\end{equation}

    

    This numerical solution can be compared with the analytical solution
above for a linear model. For this case the relation between \(\zeta\)
and \(B_1\) can be expressed as:
 
            
    
    \begin{equation}
B_{1} = 2 A_{44} \omega_{0} \zeta
\label{eq:equation}
\end{equation}

    

    and the natural frequency can be obtained from:
 
            
    
    \begin{equation}
\omega_{0} = \sqrt{\frac{C_{1}}{A_{44}}}
\label{eq:equation}
\end{equation}

    

    The analytical and numerical solutions are vary similar according to the
example: \(A_{44} = 1.0\), \(B_1 = 0.3\), \(C_1 = 5.0\) shown in the
figure below.

    \begin{Verbatim}[commandchars=\\\{\}]
findfont: Font family ['"serif"'] not found. Falling back to DejaVu Sans.
    \end{Verbatim}

    \begin{center}
    \adjustimage{max size={0.9\linewidth}{0.9\paperheight}}{output_37_1.pdf}
    \end{center}
    { \hspace*{\fill} \\}
    
    