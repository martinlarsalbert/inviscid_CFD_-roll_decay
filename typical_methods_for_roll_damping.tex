\section{Typical methods for roll
damping}\label{typical-methods-for-roll-damping}

    \subsection{Roll decay test}\label{roll-decay-test}

A common way to determine the roll damping of a ship is to conduct a
roll decay test. The initial heel angle during this test gives the ship
potential energy that subsequently is shifting to kinetic energy as the
ship starts to move during the inital phase of the roll decay test. The
energy is transfered between kinetic and potential energy during the
oscillations. The ship loses energy over time due to the damping wich
can be seen in this graph:
 
            
    
    \begin{figure}
        \begin{center}\adjustimage{max size={0.9\linewidth}{0.4\paperheight}}{figures/energy.png}\end{center}
        \caption{Energy transfer}
        \label{fig:energy}
    \end{figure}
    

    Time traces of the roll motion from roll decay tests in model scale
experiments as well as from FNPF simulations are used in this paper to
determine the roll damping of KVLCC2. Two different techniques to
identidy the damping from these tests are used: the PIT method and the
logartithmic decrement method which are both described below.

\subsection{Ikeda's method}\label{ikedas-method}

As a cheaper alternative to model tests, Ikeda's method can be used to
predict the roll damping. In this method the damping is divided into
various components. This feature enables the possibility to combine
Ikeda's method with a potential flow method. Ikeda's method is combined
with FNPF resulting in the Hybrid method as proposed in this paper.

    \subsection{PIT method}\label{pit-method}

    A parameters identification technique (PIT) similar to
\cite{7505983/EXYJELCU} is used to obtain the damping coefficients from
the roll decay model test as well as from the FNPF simulated roll decay
tests. In this technique, parameters in a mathematical model are
determined in order to get the best fit to a roll decay time signal. A
derivation of a matematical model suitable for this study is described
below together with a description of how the parameters: damping,
stiffness and inertia coefficients are determined. The roll decay motion
can be expressed in general form according to \cite{7505983/FB64RGPF}:
 
            
    
    \begin{equation}
A_{44} \ddot{\phi} + \operatorname{B_{44}}\left(\dot{\phi}\right) + \operatorname{C_{44}}\left(\phi\right) = 0
\label{eq:equation}
\end{equation}

    

    Where $B_{44}(\dot{\phi})$ and $C_{44}(\phi)$ are the damping and
stiffness models. A cubic model can be obtained by using cubic damping
and stiffness models:
 
            
    
    \begin{equation}
\operatorname{B_{44}}\left(\dot{\phi}\right) = B_{1} \dot{\phi} + B_{2} \left|{\dot{\phi}}\right| \dot{\phi} + B_{3} \dot{\phi}^{3}
\label{eq:equation}
\end{equation}

    
 
            
    
    \begin{equation}
\operatorname{C_{44}}\left(\phi\right) = C_{1} \phi + C_{3} \phi^{3} + C_{5} \phi^{5}
\label{eq:equation}
\end{equation}

    

    The total equation is then written:
 
            
    
    \begin{equation}
\begin{aligned}
A_{44} \ddot{\phi} + B_{1} \dot{\phi} + B_{2} \left|{\dot{\phi}}\right| \dot{\phi} + B_{3} \dot{\phi}^{3} + C_{1} \phi + C_{3} \phi^{3} \\ + C_{5} \phi^{5} = 0
\end{aligned}
\label{eq:equation}
\end{equation}

    

    This mathematical model can be reduced to a quadratic damping model when
$B_3=0$ and a linear model when $B_2=B_3=0$. This equation does not
have one unique solution however. If all parameters would be multiplied
by a factor $k$ these parameters would also yield as a solution to the
equation. All parameters are therefore divided by the total added mass
parameters $A_{44}$, replacing the parameters with new parameters such
as:
 
            
    
    \begin{equation}
B_{1A} = \frac{B_{1}}{A_{44}}
\label{eq:equation}
\end{equation}

    

    The equation is now rewritten with these new parameters which have
unique solutions:
 
            
    
    \begin{equation}
\begin{aligned}
B_{1A} \dot{\phi} + B_{2A} \left|{\dot{\phi}}\right| \dot{\phi} + B_{3A} \dot{\phi}^{3} + C_{1A} \phi \\ + C_{3A} \phi^{3} + C_{5A} \phi^{5} + \ddot{\phi} = 0
\end{aligned}
\label{eq:equation}
\end{equation}

    

    The parameters of this equation can be identified using least square fit
if the time signals $\phi(t)$, $\dot{\phi}(t)$ and
$\ddot{\phi}(t)$ are all known. This is the case for the results from
the FNPF simulations but not from the model tests, where only the roll
signal $\phi(t)$ is known. The other time derivatives can be estimated
using numerical differentiation of a low-pass filtered roll signal or
Kalman filtered roll signal. The filtering will however introduce some
errors in itself. So instead of using this "Differentiation approach",
it has been found that solving the differential equation numerically for
guessed parameter values determined using optimization similarly to what
was used by \cite{7505983/FJHQJJUH} and \cite{7505983/9B7QMVJJ} gives
the best parameter estimation. One problem with this "Integration
approach" is that in order to converge, the optimization needs a
resonable first guess of the parameters. The Differentiation approach
has therefore been used as a pre-step to obtain a very good first guess
of the parameters that can be passed on to the Integration approach.
This has been used for both signals from FNPF and model tests where in
the latter case numerical differentiation is used.

The differential equation is numerically solved as an intial value
problem, where the initial states for $\phi(t)$, $\dot{\phi}(t)$ and
$\ddot{\phi}(t)$ are used to estimate the following states, by
conducting very small time steps using the following expression for the
acceleration:
 
            
    
    \begin{equation}
\begin{aligned}
\ddot{\phi} = - B_{1A} \dot{\phi} - B_{2A} \left|{\dot{\phi}}\right| \dot{\phi} - B_{3A} \dot{\phi}^{3} - C_{1A} \phi - C_{3A} \phi^{3} - C_{5A} \phi^{5}
\end{aligned}
\label{eq:equation}
\end{equation}

    

    This numerical solution can be compared with an analytical solution for
a linear model (see eq. \ref{eq:linear}). For this case the relation
between $\zeta$ and $B_1$ can be expressed as:
 
            
    
    \begin{equation}
B_{1} = 2 A_{44} \omega_{0} \zeta
\label{eq:equation}
\end{equation}

    

    and the natural frequency can be obtained from:
 
            
    
    \begin{equation}
\omega_{0} = \sqrt{\frac{C_{1}}{A_{44}}}
\label{eq:equation}
\end{equation}

    

    The analytical and numerical solutions are very similar according to the
example: $A_{44} = 1.0$, $B_1 = 0.3$, $C_1 = 5.0$ shown in the
figure below.

    \begin{Verbatim}[commandchars=\\\{\}]

    \end{Verbatim}

    \begin{figure}
        \begin{center}\adjustimage{max size={0.9\linewidth}{0.4\paperheight}}{figures/analytical_numerical.pdf}\end{center}
        \caption{Analytical and numerical}
        \label{fig:analytical_numerical}
    \end{figure}
    
    