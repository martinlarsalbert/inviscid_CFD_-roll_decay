\label{abstract}

Roll motion is, however, often a critical degree of freedom that needs to be analyzed since large roll motions can result in cargo shifting or even capsizing. The viscous part of roll damping can be assessed with high accuracy by means of experimental model tests or URANS calculations, but these are generally too expensive in the early design stage of ships. Many semi-empirical formulas to
determine viscous damping were therefore developed during the 1970s, where Ikeda's method is one of the most widely used. The viscous damping from this method is normally combined with inviscid roll damping from strip theory. With today's computational power, more advanced potential flow methods can be used in the seakeeping analysis to enhance the accuracy in the predictions, but still at relatively low computational cost. This paper investigates the feasibility of combining 3D unsteady fully nonlinear potential flow (FNPF) theory solved by means of a Boundary Element Method (BEM) together with the viscous contributions from Ikeda's method. Roll decay simulations with the hybrid method proposed in this paper show very good agreement with corresponding model tests for the investigated KVLCC2 test case. Even though this approach has been investigated for such a simple case as the roll decay test and for only one ship, it is believed to be an improvement in seakeeping
potential flow simulations.

    