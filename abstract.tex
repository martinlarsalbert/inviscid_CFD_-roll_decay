%1 short intro
%What problem did you study and why is it important? Here, you want to provide some background to the study.
Many cost efficient methods have been developed over the years to analyze various aspects of ship hydrodynamics such as: resistance, propulsion and seakeeping. Getting the best possible accuracy to the lowest possible cost, is an important factor in the early design stage of ships. Using potential flow has partly presented such a solution for seakeeping, with good accuracy for heave and pitch, but not for roll, where the roll damping contains both inviscid and viscous effects. Roll motion is however often a critical degree of freedom (DOF) that needs to be analyzed since large roll motions can result in cargo shifting or even capsizing. 
%Roll motion is also very hard to predict, being the result of nonlinear restoring forces and nonlinear damping. The roll damping can be assessed with high accuracy by means of experimental model tests or RANS calculations. This is however not very efficient solutions when considering time and cost constraints, especially when being multiplied, during design iterations. Taking the shortcut of potential flow to improve efficiency with acceptable accuracy works for other DOFs such as heave or pitch, but not for roll because of large viscous effects. Many semi empirical formulas to add the viscous damping was therefore developed during the 1970s, where Ikeda's method is the most pronounced. 
The viscous part of roll damping can be assessed with high accuracy by means of experimental model tests or URANS calculations, but these are generally too expensive solutions in the early design stage of ships. Many semi empirical formulas to determine viscous damping were therefore developed during the 1970s, where Ikeda's method is one of the most widely used.
% 2 pupose/objective of study
% the motivation behind the study.
% the specific question or hypothesis you addressed. 
The viscous damping from this method is normally combined with invicid roll damping from strip theory. With today's computational power, more advanced potential flow methods can however be used in the seakeeping analysis to get better accuracy, but still at relatively low costs. 
This paper investigates the feasibility of combining 3D unsteady fully nonlinear potential flow (FNPF) solved by means of a Bounday Element Method (BEM) togther with the viscous contributions from Ikeda's method.

% 3 method/methodology that has been used
The approach of substituting the inviscid part from Ikeda's method using strip theory with FNPF is investigated by conducting roll decay simulations. Comparisons are made with both the classic strip theory approach and roll decay model tests in order to investigate potential improvements to the modelling of roll damping in potential flow seakeeping computations.

% 4 results
 


% 5 conclusions

