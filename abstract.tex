%1 short intro
%What problem did you study and why is it important? Here, you want to provide some background to the study.
For ships, roll motion is arguably the most critical degree of freedom (DOF), where very large motions can result in cargo shifting or even capsizing. Roll motion is also very hard to predict, being the result of nonlinear restoring forces and nonlinear damping. The roll damping can be assessed with high accuracy by means of experimental model tests or RANS calculations. This is however not very efficient solutions when considering time and cost constraints, especially when being multiplied, during design iterations. Taking the shortcut of potential flow to improve efficiency with acceptable accuracy works for other DOFs such as heave or pitch, but not for roll because of large viscous effects. Many semi empirical formulas to add the viscous damping was therefore developed during the 1970s, where Ikeda's method is the most pronounced.  

% 2 pupose/objective of study
% the motivation behind the study.
% the specific question or hypothesis you addressed. 
At that time, strip theory calculations was the state of the art solution to potential flow problems and it was possible to find solutions within reasonable time using computers that were then avaiable. Today, when computers are so much faster, alternatives with higher accuracy such as 3D unsteady fully nonlinear potential flow (FNPF) solved by means of a Bounday Element Method (BEM), can be used within reasonable time. FNPF together with the viscous parts of Ikeda's method is demonstrated in this paper to be a modern engineering approach to roll damping and roll motion predictions.

% 3 method/methodology that has been used
The roll damping is calculated using FNPF and Ikeda's method for the KVLCC2. The results are compared with the more classic Ikeda approach using strip theory calculations and also with results from roll decay model tests.   

% 4 results

% 5 conclusions

