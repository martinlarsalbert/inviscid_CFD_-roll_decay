%1 short intro
%What problem did you study and why is it important? Here, you want to provide some background to the study.
Many cost efficient methods have been developed over the years to analyze various aspects of ship hydrodynamics such as: resistance, propulsion and seakeeping. Getting the best possible accuracy to the lowest possible cost, is an important factor in the early design stage of ships. Using potential flow has partly presented such a solution for seakeeping, with good accuracy for heave and pitch, but not for roll, due to  roll damping viscous effects. Roll motion is however often a critical degree of freedom (DOF) that needs to be assessed since large roll motions can result in cargo shifting or even capsizing. 
%Roll motion is also very hard to predict, being the result of nonlinear restoring forces and nonlinear damping. The roll damping can be assessed with high accuracy by means of experimental model tests or RANS calculations. This is however not very efficient solutions when considering time and cost constraints, especially when being multiplied, during design iterations. Taking the shortcut of potential flow to improve efficiency with acceptable accuracy works for other DOFs such as heave or pitch, but not for roll because of large viscous effects. Many semi empirical formulas to add the viscous damping was therefore developed during the 1970s, where Ikeda's method is the most pronounced. 
The viscous roll damping can be assessed with high accuracy by means of experimental model tests or URANS calculations, but these are generally too expensive solution in the early design stage. Many semi empirical formulas to add the viscous damping to potential flow solutions was therefore developed during the 1970s, where Ikeda's method is the most pronounced. 

% 2 pupose/objective of study
% the motivation behind the study.
% the specific question or hypothesis you addressed. 
Strip theory is a simplified and extremely fast way to solve potential flow. With today's computational power, more advanced methods can however be used in the seakeeping analysis to get better accuracy with a low cost. This paper investigates 3D unsteady fully nonlinear potential flow (FNPF) solved by means of a Bounday Element Method (BEM) used together with the viscous damping from Ikeda's method to provide a modern engineering approach to roll motion analysis. 

% 3 method/methodology that has been used
The approach of combining FNPF and Ikeda's method is investigated by conducting simulations of roll decay tests. The results are compared with simulations using the more classic approach combining strip theory with Ikeda's method. Comparison is also made with results from roll decay model tests. 

% 4 results

% 5 conclusions

