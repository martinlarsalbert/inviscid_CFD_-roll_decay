%1 short intro
%What problem did you study and why is it important? Here, you want to provide some background to the study.
%For ships, roll motion is arguably the most critical degree of freedom (DOF), where very large motions can result in cargo shifting or even capsizing. Roll motion is also very hard to predict, being the result of nonlinear restoring forces and nonlinear damping. The roll damping can be assessed with high accuracy by means of experimental model tests or RANS calculations. This is however not very efficient solutions when considering time and cost constraints, especially when being multiplied, during design iterations. Taking the shortcut of potential flow to improve efficiency with acceptable accuracy works for other DOFs such as heave or pitch, but not for roll because of large viscous effects. Many semi empirical formulas to add the viscous damping was therefore developed during the 1970s, where Ikeda's method is the most pronounced. 

The effects of nonlinear restoring forces and nonlinear damping make it hard to predict roll motion accurately. The roll damping can be assessed with high accuracy by means of experimental model tests or RANS calculations, which however are not time and cost efficient, especially when multiple cases are needed for analysis during design iterations. Taking the shortcut of potential flow to improve efficiency with acceptable accuracy works for other DOFs such as heave or pitch, but not for roll because of large viscous effects. Many semi empirical formulas to add the viscous damping was therefore developed during the 1970s, where Ikeda's method is the most pronounced. 

% 2 pupose/objective of study
% the motivation behind the study.
% the specific question or hypothesis you addressed. 
In the Ikeda's method, strip theory to solve potential flow problems was possible to find solutions within reasonable time and computation capability. With rapid increase of computation power, alternative approaches such as 3D unsteady fully nonlinear potential flow (FNPF) solved by means of a Bounday Element Method (BEM), can be used to get the roll damping with higher accuracy using reasonable time effort. 

% 3 method/methodology that has been used
In this paper, FNPF together with the viscous parts of Ikeda's method is investigated to provide a modern engineering approach to roll damping and roll motion predictions. The KVLCC2 ship with experimental tests are used to verify roll damping estimation by the FNPF and Ikeda's method, through the comparison with the more classic Ikeda approach using strip theory calculations and also with results from roll decay model tests.   

% 4 results

% 5 conclusions

