\section{Introduction}\label{introduction}

The motions of a ship in a seaway can be predicted with good accuracy
using potential flow to predict the pitch and heave motions at very low
computational cost \cite{7505983/FB64RGPF}. These methods are therefore
widely used to predict ship responses during the early design stage of
ships. The roll motion is however highly influenced by viscous roll
damping, which is neglected by these methods, giving a very unrealistic
roll motion representation. But the roll motions can of course not be
dissregarded in the analysis of motions. The impact of roll motions can
be seen from the APL China casualty in 1998, where a post-Panamax C11
class container ship lost almost a third of its containers. Even newer
ships as the ONE APUS a 14,052 TEU containership built in 2019 lost more
than 1000 containers during heavy roll motions.

A lot of experimental research was conducted during the 70th and 80thout
to separate the inviscid and viscous damping \cite{7505983/937PN5DT}.
Semi empirical formulas were developed to estimate the viscous parts to
be used together with the potential flow methods. Today the linear
potential flow methods have (to a large extent) been replaced by more
advanced nonlinear methods. \cite{7505983/UGK6YEVD} has shown that the
separation of viscous and inviscous damping is still valid for a panel
method and Ikeda's method to predict the roll motion for the mentioned
APL China vessel. \cite{7505983/24TNAV5Z} has investigated an even more
advanced method, using a fully nonlinear potential flow method (FNPF)
combined with Watanabe and Inoue (WI) method to predict the viscous
damping for the DTC \cite{7505983/BYNJ8CFG} and Series60 ships. The FNPF
method is used also in the present paper, but instead of WI method,
Ikeda's method is used, as a new hybrid method. Ikeda's method is
belived to be a good method for this purpose, based on the result from
comparison with a large number of model scale roll decay tests
\cite{7505983/QMGQ76Q9}.

The implementation of the proposed hybrid method is introduced in the
next section where the underlying Ikeda's method and FNPF method are
both presented. Ikeda's method is also thurougly investigated.

    