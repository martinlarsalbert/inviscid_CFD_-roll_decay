\section{Introduction}\label{introduction}

Many great semi-empirical methods have been developed over the years to
analyze various aspects of ship hydrodynamics such as: resistance,
propulsion and seakeeping. These methods were often developed because
computing the flow numerically was far too complicated and time
consuming at the time. When computers are now faster and great
advancements in the field of Computation Fluid Dynamics (CFD) has been
achieved, the simpler semi-empirical formulas have now gotten less and
less relevant. But have they all of gotten totally irrelevant?

A semi-empirical method to predict ship roll damping, commonly known as
Ikeda's method, will be investigated in this paper, for a use case that
can still make it relevant. It will be investigated if this method can
be used to increase the accuracy of the roll motion for a fully
nonlinear potential flow method (FNPF). FNPF methods are less
computationally demanding than for example URANS methods, making them
attractive choices for seakeeping problems. The roll damping of ships is
however highly dependent on viscous effects which FNPF cannot handle. A
hybrid method is therfore proposed in this paper: adding viscous damping
from Ikeda's method to the FNPF simulations in order to improve the
accuracy of the roll motions. Similar approaches have been investigated
before: \cite{7505983/UGK6YEVD} used a panel method code and Ikeda's
method for the APL China vessel, \cite{7505983/24TNAV5Z} used FNPF
combined with Watanabe and Inoue method to predict the viscous damping.
The hybrid method in this paper is investigated using the well known
KVLCC2 test case.
 
            
    
    \begin{figure}
        \begin{center}\adjustimage{max size={0.9\linewidth}{0.4\paperheight}}{figures/body_plan.png}\end{center}
        \caption{Body plan}
        \label{fig:body_plan}
    \end{figure}
    

    This ship was selected partly because it is a well known test case and
also because it does not have any bilge keels. Ikeda's method contain
methods to predict damping from various components, where the bilge
keels is one of them. Results from roll decay simulations made with the
hybrid method will be compared to corresponding model test data from the
SSPA Maritime Dynamics Laboratory. From these model tests, only the
total damping can be observed. Reducing the number of components by
having no bilge keels will therefore give more insight into the
remaining components.

Revisiting an older semi-empirical method such as Ikeda's method can
also be used to gain a deeper understand of the roll damping
hydrodynamics.

    