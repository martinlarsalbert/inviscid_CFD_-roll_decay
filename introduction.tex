\section*{INTRODUCTION}\label{sec:introduction}
Many cost-efficient computational methods have been developed over the years to analyze various aspects of ship hydrodynamics such as:
resistance, propulsion and seakeeping. Getting the best possible
accuracy with the lowest possible computational cost is an important factor in a ship's early design stage. Potential flow-based analysis partly presents such a solution for seakeeping, with good accuracy for heave and pitch, but not for roll where the roll damping contains both inviscid and viscous effects. 
Many great semi-empirical methods have been developed over the years to analyze various aspects of ship hydrodynamics such as: resistance, propulsion and seakeeping. These methods were often developed because computing the flow numerically was far too complicated and time consuming at the time. When computers are now faster and great advancements in the field of Computation Fluid Dynamics (CFD) has been achieved, the simpler semi-empirical formulas have now gotten less and less relevant. But have they all of gotten totally irrelevant?\\


Ikeda produced many papers about various aspects of roll damping where most of them are translations from original manuscripts written in Japanese. Summaries of this method \cite{7505983/FB64RGPF}, \cite{7505983/KAKIM2E2} and \cite{7505983/UGK6YEVD} has been used
together with the original papers to understand how the method should be implemented. Falzarano says that the Himeno report and associated
computer programs are well-known to have numerous typographical errors. When looking at these resources it becomes evident that there is some variety on how the method should be implemented, regardless if this is due to typographical errors or being variations of the actual method. The implementation was therefore a fairly time consuming task where various alternative implementations needed to be compared and investigated.\\

The scale effects of roll damping is considered to mainly be associated with the skin friction component $B_F$ \cite{7505983/FB64RGPF}. This component constitute a very small part of the total damping for the full scale ship, but a substantial part for the model scale ship used in this study. This is therefore the only component in Ikedas method that needs to be recalculated when the scale changes.
For the skin friction damping $B_F$ implementation was made according to the description in \cite{7505983/UGK6YEVD}. With the difference that the actual wetted surface at rest $S$ was used instead of the proposed estimation formula.
The hull lift damping $B_L$ is calculated according to\cite{7505983/937PN5DT} and implemented as described in \cite{7505983/UYUAYY7E}. Journne added a linear interpolation to the values for $\kappa$ from the Ikeda's paper.\\

A semi-empirical method to predict ship roll damping, commonly known as Ikeda's method, will be investigated in this paper, for a use case that can still make it relevant. It will be investigated if this method can be used to increase the accuracy of the roll motion for a fully nonlinear potential flow method (FNPF). FNPF methods are less computationally demanding than for example URANS methods, making them attractive choices for seakeeping problems. The roll damping of ships is however highly dependent on viscous effects which FNPF cannot handle. A hybrid method is therfore proposed in this paper: adding viscous damping from Ikeda's method to the FNPF simulations in order to improve the accuracy of the roll motions. Similar approaches have been investigated before by: \cite{7505983/UGK6YEVD} using a panel method code and Ikeda's method for the APL China vessel, \cite{7505983/24TNAV5Z} using FNPF combined with Watanabe and Inoue method to predict the viscous damping.

    