\section{Introduction}\label{introduction}

Many great semi-empirical methods have been developed over the years to
analyze various aspects of ship hydrodynamics such as: resistance,
propulsion and seakeeping. These methods were often developed because
computing the flow numerically was far too complicated and time
consuming at the time. When computers are now faster and great
advancements in the field of Computation Fluid Dynamics (CFD) has been
achieved, the simpler semi-empirical formulas have now gotten less and
less relevant. But have they all of gotten totally irrelevant?

A semi-empirical method to predict ship roll damping, commonly known as
Ikeda's method, will be investigated in this paper, for a use case that
can still make it relevant. It will be investigated if this method can
be used to increase the accuracy of the roll motion for a fully
nonlinear potential flow method (FNPF). FNPF methods are less
computationally demanding than for example URANS methods, making them
attractive choices for seakeeping problems. The roll damping of ships is
however highly dependent on viscous effects which FNPF cannot handle. A
hybrid method is therfore proposed in this paper: adding viscous damping
from Ikeda's method to the FNPF simulations in order to improve the
accuracy of the roll motions. Similar approaches have been investigated
before by: \cite{7505983/UGK6YEVD} using a panel method code and Ikeda's
method for the APL China vessel, \cite{7505983/24TNAV5Z} using FNPF
combined with Watanabe and Inoue method to predict the viscous damping.
The hybrid method in this paper is investigated using the well known
KVLCC2 test case.

    