\section*{INTRODUCTION}\label{introduction}

Inviscid potential flow calculations can be used to solve seakeeping
problems at very low computational costs. These methods offer far
cheaper alternatives than doing for instance model tests or URANS
calculations. Potential flow calculations can therefore be used
extensively during the early design stage of ships. The pitch and heave
motions can be predicted with good accuracy, even with the older linear
strip theory methods \citep{7505983/FB64RGPF}. The roll motions will
however not be very realistic in potential flow, due to high influenced
from viscous roll damping. This is very unfortunate as the roll motions
is indeed a very important response. The impact of roll motions can be
seen from the APL China casualty in 1998, where a post-Panamax C11 class
container ship lost almost a third of its containers
\citep{7505983/WPADAQB3}. Another example is the container ship Svendborg
Maersk, were 500 containers were lost overboard and 250 containers were
damaged as a result of heavy roll motions during a passage from English
Channel to Gibraltar \citep{7505983/T78CMTDR}.

\quad A lot of experimental research was conducted during the 1970s and
80s to separate the invicid and viscous roll damping. Semi empirical
formulas were developed to estimate the viscous parts, to be used
together with the potential flow methods \citep{7505983/937PN5DT}. The
older linear methods can today be replaced by more advanced nonlinear
potential flow methods. These newer methods still need some injection of
semi empirical viscous damping to give a fair representation of the roll
motions. But is the separation of damping components still valid,
considering that these older semi empirical methods were developed in
close connection to linear strip theory? \citep{7505983/UGK6YEVD} have
shown that the separation of viscous and invicid damping is still valid
for a panel method and Ikeda's method to predict the roll motion for the
mentioned APL China vessel. \citep{7505983/24TNAV5Z} have investigated an
even more advanced method, using a fully nonlinear potential flow method
(FNPF) \citep{7505983/P4XDUMMQ} combined with Watanabe and Inoue method
(W-I) \citep{7505983/ARMIRMVY} to predict the viscous damping for the DTC
test case \citep{7505983/BYNJ8CFG}. This FNPF method is used also in the
present paper, but instead of W-I method, Ikeda's method is instead
used. Ikeda's method is believed to be a good method for the purpose,
based on results from a previous comparisons of a large number of model
scale roll decay tests \citep{7505983/QMGQ76Q9}.

\quad The implementation of the proposed hybrid method is introduced in
the next section where the underlying Ikeda's method and FNPF method are
both presented. The implementation of Ikeda's method is also closely
examined and an alternative way to calculate eddy damping is proposed.
In the validation study of the hybrid method, roll decay tests from
model tests are compared with simulations for the KVLCC2 test case. In
the roll decay tests, both decay and frequency can be observed, without
the presence of external forces from wind and waves. This gives a good
indication of the ship's roll damping and inertia. A more thorough
description of the roll decay test is given in the Roll decay test
section of this paper.

    